\documentclass{beamer}

%\usetheme{Malmoe}
\usetheme{Szeged}
\usecolortheme{crane}

\newtheorem{remark}{Remark}

\usepackage{listings}
\usepackage{beamerthemesplit}

\lstnewenvironment{newcode}{\lstset{language=Haskell,basicstyle=\scriptsize,escapechar=\@}}{}
\newcommand{\ttcode}[1]{{\color{red}{\tt{#1}}}}

\setbeamertemplate{navigation symbols}{}

\newenvironment{codeblock}[1][.8]{%
\begin{columns}
\begin{column}{#1\linewidth}
\begin{exampleblock}{}}{%
\end{exampleblock}
\end{column}
\end{columns}} 

\newenvironment{execblock}[1][.8]{%
\begin{columns}
\begin{column}{#1\linewidth}
\begin{block}{}}{%
\end{block}
\end{column}
\end{columns}} 

\def\slideskip{\vskip 0.1in}
\def\frameskip{\vskip 0.1in}

\usepackage{hyperref}

\newcommand{\forget}[1]{}
\title{Haskell Boot Camp}
\subtitle{CS4430 Spring 2017}
\author{Bill Harrison}
\date{\today}

\begin{document}

\frame{\titlepage}

\forget{
\section[Outline]{}
\frame{\tableofcontents}
\section{Introduction}
%\subsection{Overview of the Beamer Class}
\frame
{
  \frametitle{Features of the Beamer Class}

  \begin{itemize}
  \item<1-> Normal LaTeX class.
  \item<2-> Easy overlays.
  \item<3-> No external programs needed.      
  \end{itemize}
}
}

%%%%%%%%%%%%%%%%%%%%%%%%%%%%
%%%%%%%%%%%%%%%%%%%%%%%%%%%%
%%%%%%%%%%%%%%%%%%%%%%%%%%%%
\frame
{
    \frametitle{Haskell Basics}
\begin{itemize}
\item Modern (pure) lazy functional language
\item Statically typed, supports type inference
\item Compilers and interpreters:
\begin{itemize}
\item http://www.haskell.org/implementations.html
\item Hugs interpreter
\item \textcolor{red}{GHC Compiler} Haskell Platform: \url{https://www.haskell.org/platform/}
\end{itemize}
\item A peculiar language feature: indentation matters
\item Also: capitalization matters
\end{itemize}
}


\begin{frame}[fragile]
\frametitle{Reading}

Read the following:
\begin{itemize}
\item Learn You a Haskell: Chapters 1 \& 2

\item Gentle Introduction to Haskell ({\scriptsize{\color{red}{\url{http://www.haskell.org/tutorial/}}}}): Sections 1 \& 2
\end{itemize}
\end{frame}


\begin{frame}[fragile]
\frametitle{Data + Algorithms = Programs}
\begin{itemize}

\item Any program is a combination of {\color{red}{data structures}} and {\color{blue}{code}} that  {\color{green}{manipulates}} that data

\pause
\item Ex: simple arithmetic interpreter
	\begin{itemize}
    \item {\color{red}{data structure}}: \verb+data Exp = Const Int | Neg Exp | Add Exp Exp+
    \item {\color{blue}{code}}:
    \begin{verbatim}
    interp :: Exp -> Int
    interp (Const i)   = i
    interp (Neg e)     = - (interp e)
    interp (Add e1 e2) = interp e1 + interp e2
    \end{verbatim}
	\end{itemize}
\pause

\item {\color{green}{Manipulation}}: How do Haskell programs use data?
\begin{itemize}
\item Patterns break data apart to access: \\``{\small{\tt interp {\color{red}{(Neg e)}} =$\ldots$}}''
\item Functions recombine into new data: \\``{\small{\tt interp e1 {\bf{\color{red}{+}}} interp e2}}''
\end{itemize}

\end{itemize}
\end{frame}


\begin{frame}[fragile]
\Large

\frametitle{Type Declarations}

In Haskell, a new name for an existing type can be 
defined using a \underline{type declaration}. 

\slideskip

\begin{codeblock}
\vspace{-2ex}
\begin{verbatim}
type String = [Char] 
\end{verbatim}
\vspace{-2ex}
\end{codeblock}

\slideskip

{\tt String} is a synonym for the type \verb+[Char]+.
\end{frame}


\begin{frame}[fragile]
\Large

%\frametitle{Type Declarations}

Type declarations can be used to make other types 
easier to read.  
For example, given 

\slideskip

\begin{codeblock}
\vspace{-2ex}
\begin{verbatim}
type Pos = (Int,Int)
\end{verbatim}
\vspace{-2ex}
\end{codeblock}

\slideskip
we can define
\slideskip
\begin{codeblock}
\vspace{-2ex}
\begin{verbatim}
origin    :: Pos 
origin     = (0,0) 

left      :: Pos -> Pos 
left (x,y) = (x-1,y) 
\end{verbatim}
\vspace{-2ex}
\end{codeblock}

\end{frame}

\begin{frame}[fragile]
\Large

Like function definitions, type declarations can also 
have \underline{parameters}. 
For example, given 

\slideskip

\begin{codeblock}
\vspace{-2ex}
\begin{verbatim}
type Pair a = (a,a) 
\end{verbatim}
\vspace{-2ex}
\end{codeblock}

\slideskip
we can define
\slideskip

\begin{codeblock}
\vspace{-2ex}
\begin{verbatim}
mult      :: Pair Int -> Int 
mult (m,n) = m*n

copy      :: a -> Pair a 
copy x     = (x,x)
\end{verbatim}
\vspace{-2ex}
\end{codeblock}

\end{frame}

\begin{frame}[fragile]
\Large

Type declarations can be nested:

\slideskip

\begin{codeblock}
\vspace{-2ex}
\begin{verbatim}
type Pos   = (Int,Int)    -- GOOD

type Trans = Pos -> Pos   -- GOOD
\end{verbatim}
\vspace{-2ex}
\end{codeblock}

\slideskip
However, they cannot be recursive:
\slideskip

\begin{codeblock}
\vspace{-2ex}
\begin{verbatim}
type Tree = (Int,[Tree])  -- BAD
\end{verbatim}
\vspace{-2ex}
\end{codeblock}

\end{frame}

\begin{frame}[fragile]
\LARGE

\frametitle{Data Declarations}

A completely new type can be defined by specifying 
its values using a \underline{data declaration}. 

\slideskip

\begin{codeblock}
\vspace{-2ex}
\begin{verbatim}
data Bool = False | True
\end{verbatim}
\vspace{-2ex}
\end{codeblock}

\slideskip
{\tt Bool} is a new type, with two 
new values {\tt False} and {\tt True}.

\end{frame}

\begin{frame}[fragile]
\LARGE

Note: 
\begin{itemize}
\item The two values False and True are called the 
constructors for the type Bool. 
\item Type and constructor names must begin with 
an upper-case letter. 
\item Data declarations are similar to context free 
grammars.  The former specifies the values of 
a type, the latter the sentences of a language.
\end{itemize}
\end{frame}


\begin{frame}[fragile]
\large

Values of new types can be used in the same ways 
as those of built in types.  For example, given
\slideskip

\begin{codeblock}
\vspace{-2ex}
\begin{verbatim}
data Answer = Yes | No | Unknown 
\end{verbatim}
\vspace{-2ex}
\end{codeblock}

\slideskip
we can define:
\slideskip

\begin{codeblock}
\vspace{-2ex}
\begin{verbatim}
answers     :: [Answer] 
answers      = [Yes,No,Unknown] 

flip        :: Answer -> Answer 
flip Yes     = No 
flip No      = Yes 
flip Unknown = Unknown 
\end{verbatim}
\vspace{-2ex}
\end{codeblock}

\end{frame}

\begin{frame}[fragile]
\large

The constructors in a data declaration can also have 
parameters.  For example, given 
\slideskip

\begin{codeblock}
\vspace{-2ex}
\begin{verbatim}
data Shape = Circle Float 
           | Rect Float Float 
\end{verbatim}
\vspace{-2ex}
\end{codeblock}

\slideskip
we can define:
\slideskip

\begin{codeblock}
\vspace{-2ex}
\begin{verbatim}
square         :: Float -> Shape 
square n        = Rect n n 
area           :: Shape -> Float 
area (Circle r) = pi * r^2 
area (Rect x y) = x * y 
\end{verbatim}
\vspace{-2ex}
\end{codeblock}

\end{frame}

\begin{frame}[fragile]
%\LARGE

Note: 
\begin{itemize}
\item Shape has values of the form Circle r where r is 
a float, and Rect x y where x and y are floats. 
\item Circle and Rect can be viewed as functions that 
construct values of type Shape: 
\end{itemize}


\begin{execblock}[1.0]
\vspace{-2ex}
\begin{verbatim}
-- Not a definition
Circle :: Float -> Shape 
Rect   :: Float -> Float -> Shape
\end{verbatim}
\vspace{-2ex}
\end{execblock}


\end{frame}



\begin{frame}[fragile]
\large

Not surprisingly, data declarations themselves can 
also have parameters.  For example, given 
\slideskip

\begin{codeblock}
\vspace{-2ex}
\begin{verbatim}
data Maybe a = Nothing | Just a 
\end{verbatim}
\vspace{-2ex}
\end{codeblock}

\slideskip
we can define:
\slideskip

\begin{codeblock}
\vspace{-2ex}
\begin{verbatim}
safediv    :: Int -> Int -> Maybe Int 
safediv _ 0 = Nothing 
safediv m n = Just (m `div` n) 

safehead   :: [a] -> Maybe a 
safehead [] = Nothing 
safehead xs = Just (head xs) 
\end{verbatim}
\vspace{-2ex}
\end{codeblock}

\end{frame}

\begin{frame}[fragile]
\Large

\frametitle{Recursive Types}

In Haskell, new types can be declared in terms of 
themselves.  That is, types can be \underline{recursive}. 

\slideskip

\begin{codeblock}
\vspace{-2ex}
\begin{verbatim}
data Nat = Zero | Succ Nat
\end{verbatim}
\vspace{-2ex}
\end{codeblock}

\slideskip
Nat is a new type, with constructors 
{\tt Zero :: Nat} and {\tt Succ :: Nat -> Nat}.

\end{frame}

\begin{frame}[fragile]

Note: 
\begin{itemize}
\item A value of type {\tt Nat} is either {\tt Zero}, or of the form 
{\tt Succ n} where {n :: Nat}.  That is, {\tt Nat} contains the 
following infinite sequence of values: 
\end{itemize}

\begin{execblock}
\vspace{-2ex}
\begin{verbatim}
Zero
Succ Zero
Succ (Succ Zero)
\end{verbatim}
\vspace{-2ex}
\hskip 1in {\Huge$\vdots$}
\end{execblock}


\end{frame}


\begin{frame}[fragile]

Note: 
\begin{itemize}
\item We can think of values of type Nat as natural 
numbers, where {\tt Zero} represents {\tt 0}, and {\tt Succ} 
represents the successor function {\tt 1+}. 
\item For example, the value
\begin{execblock}
\vspace{-1ex}
\begin{verbatim}
Succ (Succ (Succ Zero))
\end{verbatim}
\vspace{-1ex}
\end{execblock}

represents the natural number
\begin{execblock}
\vspace{-1ex}
\begin{verbatim}
1 + (1 + (1 + 0))
\end{verbatim}
\vspace{-1ex}
\end{execblock}

\end{itemize}

\end{frame}



\begin{frame}[fragile]

\large
Using recursion, it is easy to define functions that 
convert between values of type Nat and Int: 

\frameskip

\begin{codeblock}[1.0]
\vspace{-2ex}
\begin{verbatim}
nat2int         :: Nat -> Int 
nat2int Zero     = 0 
nat2int (Succ n) = 1 + nat2int n 

int2nat         :: Int -> Nat 
int2nat 0        = Zero 
int2nat n        = Succ (int2nat (n - 1))
\end{verbatim}
\vspace{-2ex}
\end{codeblock}

\end{frame}

\begin{frame}[fragile]

Two naturals can be added by converting them to 
integers, adding, and then converting back: 

\frameskip

\begin{codeblock}
\vspace{-2ex}
\begin{verbatim}
add    :: Nat -> Nat -> Nat 
add m n = int2nat (nat2int m + nat2int n) 
\end{verbatim}
\vspace{-2ex}
\end{codeblock}
\frameskip
However, using recursion the function add can be 
defined without the need for conversions: 
\frameskip
\begin{codeblock}
\vspace{-2ex}
\begin{verbatim}
add Zero     n = n 
add (Succ m) n = Succ (add m n) 
\end{verbatim}
\vspace{-2ex}
\end{codeblock}
\frameskip
The recursive definition for add corresponds to 
the laws 
$$0+n = n$$ and $$(1+m)+n = 1+(m+n)$$
\end{frame}

\begin{frame}[fragile]

\Large
Using recursion, an expression tree can be defined using:

\frameskip

\begin{codeblock}
\vspace{-2ex}
\begin{verbatim}
data Expr = Val Int 
          | Add Expr Expr 
          | Mul Expr Expr 
\end{verbatim}
\vspace{-2ex}
\end{codeblock}
\frameskip

One example of such a tree written in Haskell is
\frameskip
\begin{execblock}
\begin{verbatim}
Add (Val 1) (Mul (Val 2) (Val 3))
\end{verbatim}
\end{execblock}
\frameskip


\end{frame}


\begin{frame}[fragile]

\Large
Using recursion, it is now easy to define functions 
that process expressions.  For example: 

\frameskip

\begin{codeblock}
\begin{verbatim}
size          :: Expr -> Int 
size (Val n)   = 1 
size (Add x y) = size x + size y 
size (Mul x y) = size x + size y 

eval          :: Expr -> Int 
eval (Val n)   = n 
eval (Add x y) = eval x + eval y 
eval (Mul x y) = eval x * eval y
\end{verbatim}
\end{codeblock}


\end{frame}

\begin{frame}[fragile]
\LARGE

Note: 
\begin{itemize}
\item 
The three constructors have types: 

\begin{execblock}[1.0]
\begin{verbatim}
-- Not a definition
Val :: Int -> Expr 
Add :: Expr -> Expr -> Expr 
Mul :: Expr -> Expr -> Expr 
\end{verbatim}
\end{execblock}

\end{itemize}

\end{frame}


\begin{frame}[fragile]

\Large
Using recursion, a binary tree can be defined using:

\frameskip

\begin{codeblock}
\begin{verbatim}
data Tree = Leaf Int 
          | Node Tree Int Tree 
\end{verbatim}
\end{codeblock}
\frameskip

One example of such a tree written in Haskell is
\frameskip
\begin{execblock}
\begin{verbatim}
Node (Node (Leaf 1) 3 (Leaf 4)) 
     5 
     (Node (Leaf 6) 7 (Leaf 9))
\end{verbatim}
\end{execblock}
\frameskip
\end{frame}


\begin{frame}[fragile]
\large

We can now define a function that decides if a given 
integer occurs in a binary tree: 

\frameskip

\begin{execblock}[1.0]
\begin{verbatim}
occurs               :: Int -> Tree -> Bool 
occurs m (Leaf n)     = m==n 
occurs m (Node l n r) = m==n 
                        || occurs m l 
                        || occurs m r 

-- N.b., || is ``or''
\end{verbatim}
\end{execblock}

\frameskip

In the worst case, when the integer does not 
occur, this function traverses the entire tree.

\end{frame}

\begin{frame}[fragile]
\large

Search trees have the important property that when 
trying to find a value in a tree we can always decide 
which of the two sub-trees it may occur in: 

\frameskip

\begin{execblock}[1.0]
\begin{verbatim}
occurs               :: Int -> Tree -> Bool 
occurs m (Leaf n)            = m==n 
occurs m (Node l n r) | m==n = True 
                      | m<n  = occurs m l 
                      | m>n  = occurs m r
                      \end{verbatim}
\end{execblock}

\frameskip

This new definition is more \underline{efficient}, because it only 
traverses one path down the tree. 

What are we assuming at each {\tt Node}?

\end{frame}
\begin{frame}[fragile]
\large

Finally consider the function {\tt flatten} that returns the 
list of all the integers contained in a tree: 

\frameskip

\begin{execblock}
\begin{verbatim}
flatten             :: Tree -> [Int] 
flatten (Leaf n)     = [n] 
flatten (Node l n r) = flatten l 
                       ++ [n] 
                       ++ flatten r 
\end{verbatim}
\end{execblock}

\frameskip
\frameskip

A tree is a \underline{search tree} if it flattens to a list that is ordered.
\end{frame}
%%%%%%%%%%%%%%%%%%%%%%%%%%%%
%%%%%%%%%%%%%%%%%%%%%%%%%%%%
%%%%%%%%%%%%%%%%%%%%%%%%%%%%
\begin{frame}[fragile]
    \frametitle{Type inference}
\begin{itemize}
\item 
{\color{red}\verb^x = 1 + 2^}
\item[] {\color{red}\verb^1^} has type {\color{red}\verb^Integer^}, {\color{red}\verb^2^} has type {\color{red}\verb^Integer^}, adding two Integers
%
%\item[]
results in another \ttcode{Integer}, therefore  \ttcode{x :: Integer}. \footnote{Actually, member of {\tt Num} type class  is inferred; but, {\tt Integer} $\in$ {\tt Num}.}

\pause
\item
{\color{red}\verb^inc x = x + 1^}
With similar reasoning, {\color{red}\verb^inc :: Integer -> Integer^}
\pause

\item Explicit type annotations are possible:
{\color{red}\begin{verbatim}
inc :: Integer -> Integer
inc x = x + 1
\end{verbatim}
}

\end{itemize}
\end{frame}

%%%%%%%%%%%%%%%%%%%%%%%%%%%%
%%%%%%%%%%%%%%%%%%%%%%%%%%%%
%%%%%%%%%%%%%%%%%%%%%%%%%%%%
\begin{frame}[fragile]
    \frametitle{Lists in Haskell}
\begin{itemize}
\item The data-structure for almost everything is List

\item Constructing lists:
{\color{red}
\begin{newcode}
[]        -- @empty list@
[1]       -- @list with one element@
[1, 2, 3] -- @a longer list@
\end{newcode}
}
\pause

\item List patterns:
\begin{itemize}
\item \ttcode{x:xs} matches to any list with one or more elements
\item \ttcode{x:y:z:xs} matches to any list with three or more elements
\item \ttcode{[x]} matches to any list with one element
\item \ttcode{[]} matches to empty list
\end{itemize}
\pause

\item[]

{\color{red}
\begin{newcode}
let x:xs = [1, 2, 3]
 -- x is 1
 -- xs is [2, 3]
\end{newcode}
}

\end{itemize}


\end{frame}


%%%%%%%%%%%%%%%%%%%%%%%%%%%%
%%%%%%%%%%%%%%%%%%%%%%%%%%%%
%%%%%%%%%%%%%%%%%%%%%%%%%%%%
\begin{frame}[fragile]
    \frametitle{Defining Functions}
\begin{itemize}

\item Defined as equations (with pattern matching)
{\color{red}
\begin{newcode}
len1::[a] -> Integer
len1 [] = 0
len1 (x:xs) = 1 + len1 xs
\end{newcode}
}
\pause

\item With lambda abstraction
{\color{red}
\begin{newcode}
len2::[a] -> Integer
len2 = \ x -> if (null x) then 0 else 1 + (len2 (tail x))
\end{newcode}
}

\pause
\item Note the function invocation syntax:
{\color{red}
\begin{newcode}
(len1 [1, 2, 3])
\end{newcode}
}

\end{itemize}
\end{frame}


%%%%%%%%%%%%%%%%%%%%%%%%%%%%
%%%%%%%%%%%%%%%%%%%%%%%%%%%%
%%%%%%%%%%%%%%%%%%%%%%%%%%%%
\begin{frame}[fragile]
    \frametitle{Haskell functions can be \emph{curried}}
{\large
    {\color{red}
    \begin{newcode}
    add::Int -> Int -> Int
    add x y = x + y
    
    add3::Int -> Int
    add3 = add 3
    
    z::Int
    z = add3 4
    \end{newcode}
    }
}
\begin{remark}[Currying relies on the following isomorphism:]
\(A \rightarrow B \rightarrow C  \cong (A \times B) \rightarrow C \)
\end{remark}

\end{frame}

%%%%%%%%%%%%%%%%%%%%%%%%%%%%
%%%%%%%%%%%%%%%%%%%%%%%%%%%%
%%%%%%%%%%%%%%%%%%%%%%%%%%%%
\begin{frame}[fragile]
    \frametitle{Haskell is \emph{pure}}
\begin{itemize}
\item I.e., no side effects (e.g. assignments, etc.). For example, in 
{\color{red}
\begin{verbatim}
x = add 1 2 
\end{verbatim}
}
\begin{itemize}
\item a fresh variable \ttcode{x} is bound to the value of \ttcode{add 1 2},
\item the value of \ttcode{add 1 2} is not computed until the value of \ttcode{x} is required (\emph{lazy evaluation}), 
\item \ttcode{x} stays bound to \ttcode{add 1 2}  within the scope of definition. 
\end{itemize}

\pause
\item \(\therefore\) Haskell functions are pure "mathematical" functions 
\begin{itemize}
\item Makes reasoning about programs feasible 
N.b., side-effects are necessary for realistic programming (for 
IO, efficiency, ...). 

\item Haskell type system encapsulates all effects inside 
\emph{monads} 

\end{itemize}
\end{itemize}
\end{frame}


%%%%%%%%%%%%%%%%%%%%%%%%%%%%
%%%%%%%%%%%%%%%%%%%%%%%%%%%%
%%%%%%%%%%%%%%%%%%%%%%%%%%%%
\begin{frame}[fragile]
    \frametitle{Haskell is \emph{lazy}}
\begin{itemize}
\item Lazy evaluation (a.k.a., call-by-need): 
Never evaluate an expression, unless it�s value is needed 

\item
Example: The following program is not erroneous. 
{\color{red}
\begin{verbatim}
omit x = 0 
v      = omit (1/0) 
main   = putStr (show v) 
\end{verbatim}
}
\end{itemize}
\end{frame}


%%%%%%%%%%%%%%%%%%%%%%%%%%%%
%%%%%%%%%%%%%%%%%%%%%%%%%%%%
%%%%%%%%%%%%%%%%%%%%%%%%%%%%
\begin{frame}[fragile]
    \frametitle{Parametric Polymorphism}
\begin{itemize}
\item Examples: 
{\color{red}
\begin{verbatim}
id :: a -> a 
id x = x 

length :: [a] -> Int 
length []     = 0 
length (x:xs) = 1 + length xs
 
tail :: [a] -> [a] 
tail []     = [] 
tail (x:xs) = xs 

eval::(a -> b) -> a -> b 
eval f x = f x 
\end{verbatim}
}
\item Note syntax for type parameters 
\end{itemize}
\end{frame}




%%%%%%%%%%%%%%%%%%%%%%%%%%%%
%%%%%%%%%%%%%%%%%%%%%%%%%%%%
%%%%%%%%%%%%%%%%%%%%%%%%%%%%
\begin{frame}[fragile]
    \frametitle{Type Classes}
\begin{itemize}
\item Consider now a non-parameterically polymorphic function. 
{\color{red}
\begin{verbatim}
not_equal:: a -> a -> Bool ??? 
not_equal x y = if (x == y) then False else True 
\end{verbatim}
}

\pause
\item There are requirements for \ttcode{a};  
Not all \ttcode{a}'s will be acceptable. 

\pause
\item The type bound to \ttcode{a} must be \emph{equality 
comparable}

\pause
\item
\ttcode{a} must be an instance of the type class \ttcode{Eq}
{\color{red}
\begin{verbatim}
not_equal:: Eq a => a -> a -> Bool 
not_equal x y = if (x == y) then False else True 
\end{verbatim}
}

\end{itemize}
\end{frame}

%%%%%%%%%%%%%%%%%%%%%%%%%%%%
%%%%%%%%%%%%%%%%%%%%%%%%%%%%
%%%%%%%%%%%%%%%%%%%%%%%%%%%%
\begin{frame}[fragile]
    \frametitle{Motivating Type Classes}
\begin{itemize}
\item Primary motivation: Function overloading mechanism for Haskell 
(ad-hoc polymorphism)\footnote{Wadler, Blott: "How to Make Ad-Hoc Polymorphism Less Ad Hoc", 1988} 
\begin{itemize}
\item Overloading with type classes is akin to OO overloading
\end{itemize}
\item Two different kinds of polymorphism in Haskell
\begin{itemize}
\item Parametric polymorphism: one implementation covers all types 
\item Ad-hoc polymorphism: same syntax for different implementations 
\end{itemize}

\end{itemize}
\end{frame}


%%%%%%%%%%%%%%%%%%%%%%%%%%%%
%%%%%%%%%%%%%%%%%%%%%%%%%%%%
%%%%%%%%%%%%%%%%%%%%%%%%%%%%
\begin{frame}[fragile]
    \frametitle{Type Classes (cont'd)}
\begin{itemize}
\item Type classes represent a set of requirements 
\item Requirements are expressed as function signatures 
\item Default implementations for each signature can be provided 
\item Example: 
{\color{red}
\begin{verbatim}
class Eq a where 
   (==), (/=) :: a -> a -> Bool 
\end{verbatim}
}
%   x /= y     = not (x == y) 
%   x == y     = not (x /= y) 

\item 
The class definition can be read as: 
{\it A class of types that conforms to the specified interface}

\item
Note how the declaration of conformance is separate from the 
definition of a type (unlike, say, \ttcode{implements} in Java) 
\end{itemize}
\end{frame}



%%%%%%%%%%%%%%%%%%%%%%%%%%%%
%%%%%%%%%%%%%%%%%%%%%%%%%%%%
%%%%%%%%%%%%%%%%%%%%%%%%%%%%
\begin{frame}[fragile]
    \frametitle{Instances of Type Classes}
\begin{itemize}
\item Members of type classes are called \emph{instances}. 
A type is not an instance of a type class unless explicitly defined as such:
{\color{red}
\begin{verbatim} 
instance Eq Bool where 
   True == True   = True 
   False == False = True 
   _ == _         = False 
\end{verbatim}
}

\item
This would be painful without parameterized instance declarations, 
referred to as "conditional instance declarations". Example: 
{\color{red}
\begin{verbatim}
instance Eq a => Eq [a] where 
    [] == []         = True 
    (x:xs) == (y:ys) = x==y && xs==ys 
     _ == _          = False 
\end{verbatim}
}

\item
\ttcode{Eq a =>} is the context (constraint). 

\end{itemize}

\end{frame}

%%%%%%%%%%%%%%%%%%%%%%%%%%%%
%%%%%%%%%%%%%%%%%%%%%%%%%%%%
%%%%%%%%%%%%%%%%%%%%%%%%%%%%
\begin{frame}[fragile]
    \frametitle{Constraining polymorphic functions}
\begin{itemize}
\item
If a function is not explicitly annotated with its type, constraints will 
be deduced with type inference 
{\color{red}
\begin{verbatim}
not_equal x y = if (x == y) then False else True 
\end{verbatim}
}

\item
From \ttcode{x == y} it can be inferred that the types of \ttcode{x} and \ttcode{y} must be 
instances of \ttcode{Eq}, and they must be of the same type. 

\item
The type of \verb@not_equal@ is thus deduced to: 
{\color{red}
\begin{verbatim}
not_equal :: Eq a => a -> a -> Bool 
\end{verbatim}
}

\item
Type inference determines the least constrained function type 
(a.k.a., principal type). 

\item Type annotations are an important form of documentation
\begin{itemize}
\item annotations are (usually) not essential
\item sometimes must to help the type inference process (polymorphic recursion) 
\end{itemize}

\item Consequence of type inference: a particular function name, such as \ttcode{==} 
can only be required by one type class.
\end{itemize}
\end{frame}





%%%%%%%%%%%%%%%%%%%%%%%%%%%%
%%%%%%%%%%%%%%%%%%%%%%%%%%%%
%%%%%%%%%%%%%%%%%%%%%%%%%%%%
\begin{frame}[fragile]
    \frametitle{Inheritance in type classes }
\begin{itemize}
\item
$\ldots$ is comparable to extending interfaces in 
Java 

\item
Accomplished with conditional class de�nitions. 

\item
The same syntax Eq a for expressing the context is used. 
{\color{red}
\begin{verbatim}
class Eq a => Ord a where 
   (<), (<=), (>), (>=) :: a -> a -> Bool 
   max, min             :: a -> a -> a 
   compare              :: a -> a -> Ordering 
\end{verbatim}
}

\item
To be an instance of \ttcode{Ord}, type must meet the signature requirements 
listed in \ttcode{Ord} and those of \ttcode{Eq}. 

\item
An instance declaration that makes a type an instance of \ttcode{Ord} does not 
establish that the type is an instance of \ttcode{Eq}! 

\end{itemize}
\end{frame}


%%%%%%%%%%%%%%%%%%%%%%%%%%%%
%%%%%%%%%%%%%%%%%%%%%%%%%%%%
%%%%%%%%%%%%%%%%%%%%%%%%%%%%
\begin{frame}[fragile]
    \frametitle{Multiple type class constraints}
\begin{itemize}
\item 
A single type parameter can be constrained with several type classes. 

\item
E.g. a function that needs to compare values, and also show them as 
strings: 
{\color{red}
\begin{verbatim}
class Show a where 
    show     :: a -> String 
    show_min :: (Ord a, Show a) => a -> a -> String 
    show_min x y = show (min x y) 
\end{verbatim}
}

\end{itemize}
\end{frame}



%%%%%%%%%%%%%%%%%%%%%%%%%%%%
%%%%%%%%%%%%%%%%%%%%%%%%%%%%
%%%%%%%%%%%%%%%%%%%%%%%%%%%%
\begin{frame}[fragile]
    \frametitle{Modules, Data Types, Libraries}
\begin{itemize}
\item \ttcode{data} vs. \ttcode{newtype} vs. \ttcode{type}

\item records, tuples, lists 

\item \ttcode{import}

\end{itemize}
\end{frame}



\end{document}
    
