\documentclass{beamer}

%\usetheme{Malmoe}
\usetheme{Szeged}
\usecolortheme{crane}

\newtheorem{remark}{}
\setbeamertemplate{navigation symbols}{}

\usepackage{listings}
\renewcommand{\ttdefault}{pcr}
\definecolor{gray_ulisses}{gray}{0.55}
\definecolor{castanho_ulisses}{rgb}{0.71,0.33,0.14}
\definecolor{preto_ulisses}{rgb}{0.41,0.20,0.04}
\definecolor{green_ulises}{rgb}{0.2,0.75,0}
\lstdefinelanguage{HaskellUlisses} {
        basicstyle=\itfamily\small,
        sensitive=true,
        morecomment=[l][\ttfamily\tiny]{--},
        morecomment=[s][\ttfamily\tiny]{\{-}{-\}},
%       morecomment=[l][\color{gray_ulisses}\ttfamily]{--},
%       morecomment=[s][\color{gray_ulisses}\ttfamily]{\{-}{-\}},
%       morecomment=[l][\color{gray_ulisses}\ttfamily\tiny]{--},
%       morecomment=[s][\color{gray_ulisses}\ttfamily\tiny]{\{-}{-\}},
        morestring=[b]",
        %stringstyle=\color{red},
        showstringspaces=false,
        numberstyle=\tiny,
        numberblanklines=true,
        showspaces=false,
        breaklines=true,
        showtabs=false,
        emph=
        {[1]
                        FilePath,IOError,abs,acos,acosh,all,and,any,appendFile,approxRational,asTypeOf,asin,
                asinh,atan,atan2,atanh,basicIORun,break,catch,ceiling,chr,compare,concat,concatMap,
                const,cos,cosh,curry,cycle,decodeFloat,denominator,digitToInt,div,divMod,drop,
                dropWhile,either,elem,encodeFloat,enumFrom,enumFromThen,enumFromThenTo,enumFromTo,
                error,even,exp,exponent,fail,filter,flip,floatDigits,floatRadix,floatRange,floor,
                fmap,foldl,foldl1,foldr,foldr1,fromDouble,fromEnum,fromInt,fromInteger,fromIntegral,
                fromRational,gcd,getChar,getContents,getLine,head,id,inRange,index,init,intToDigit,
                interact,ioError,isAlpha,isAlphaNum,isAscii,isControl,isDenormalized,isDigit,isHexDigit,
                isIEEE,isInfinite,isLower,isNaN,isNegativeZero,isOctDigit,isPrint,isSpace,isUpper,iterate,iter,iterS,
                last,lcm,length,lex,lexDigits,lexLitChar,lines,log,logBase,lookup,map,mapM,mapM_,max,
                maxBound,maximum,maybe,min,minBound,minimum,mod,negate,not,notElem,null,numerator,odd,
                or,ord,pi,pred,primExitWith,print,product,properFraction,putChar,putStr,putStrLn,quot,
                quotRem,range,rangeSize,read,readDec,readFile,readFloat,readHex,readIO,readInt,readList,readLitChar,
                readLn,readOct,readParen,readSigned,reads,readsPrec,realToFrac,recip,refold,rem,replicate,
                reverse,round,scaleFloat,scanl,scanl1,scanr,scanr1,seq,sequence,sequence_,show,showChar,showInt,
                showList,showLitChar,showParen,showSigned,showString,shows,showsPrec,significand,signum,sin,
                sinh,snd,span,splitAt,sqrt,subtract,succ,sum,tail,take,takeWhile,tan,tanh,threadToIOResult,toEnum,
                toInt,toInteger,toLower,toRational,toUpper,truncate,uncurry,unlines,until,unwords,unzip,
                unzip3,userError,words,writeFile,zip,zip3,zipWith,zipWith3,listArray,doParse,lift,signal,get,put,extrude,return
        },
        emphstyle={[1]\textbf},
        emph=
        {[2]
                Double,Either,IO,Ordering,Rational,Ratio,ReadS,ShowS,
                Word8,InPacket,ReT,StT,I
        },
        emphstyle={[2]\textbf},
        emph=
        {[3]
                case,class,deriving,do,else,if,import,in,infixl,infixr,instance,let,
                module,of,primitive,then,type,vhdl,newtype
        },
        emphstyle={[3]\textbf},
        emph=
        {[4]
                quot,rem,div,mod,elem,notElem,seq
        },
%        emphstyle={[4]\color{castanho_ulisses}\textbf},
        emphstyle={[4]\textbf},
        emph=
        {[5]
                EQ,False,GT,Just,LT,Left,Nothing,Right,Show,Eq,Ord,Num
        },
%        emphstyle={[5]\color{preto_ulisses}\textbf}
        emphstyle={[5]\textbf}
}

\lstnewenvironment{footcode}
}
{}
%{\hrulesmallskip} not sure what this is.

%\lstnewenvironment{hcode}
%{%\textbf{Haskell Code} \hspace{1cm} \hrulefill
% \lstset{language=HaskellUlisses,    basicstyle=\ttfamily\small,escapechar=\%}}
%{}

\lstnewenvironment{hcode}
}
{}

\lstnewenvironment{smallcode}
{%\hspace{1cm}
 \lstset{language=HaskellUlisses,       basicstyle=\ttfamily\small,escapechar=\%
}}
{}
%{\hrulesmallskip} not sure what this is.

\lstnewenvironment{tinycode}
{%\hspace{1cm}
 \lstset{language=HaskellUlisses,       basicstyle=\ttfamily\scriptsize,escapechar=\%
}}
{}

\lstnewenvironment{teenycode}
{%\hspace{1cm}
 \lstset{language=HaskellUlisses,       basicstyle=\ttfamily\tiny,escapechar=\%
}}
{}


\lstnewenvironment{newcode}{\lstset{language=Haskell,basicstyle=\scriptsize,identifierstyle=\itshape,stringstyle=\itshape,escapechar=\%}}{}

\lstnewenvironment{normalcode}{\lstset{language=Haskell,
                                       basicstyle=\normalsize\ttfamily,
                                       keywordstyle=\bfseries,
                                       %identifierstyle=\itshape,
                                       %stringstyle=\itshape,
                                       escapechar=\%}}{}

\lstnewenvironment{vhdl}
{\lstset{language=VHDL,basicstyle=\ttfamily\small,escapechar=\%}}{}

\lstdefinelanguage{PreHDL} {
        basicstyle=\ttfamily\small,
        sensitive=true,
        morecomment=[l][\ttfamily\tiny]{//},
%        morecomment=[s][\ttfamily\tiny]{\{-}{-\}},
%       morecomment=[l][\color{gray_ulisses}\ttfamily]{--},
%       morecomment=[s][\color{gray_ulisses}\ttfamily]{\{-}{-\}},
%       morecomment=[l][\color{gray_ulisses}\ttfamily\tiny]{--},
%       morecomment=[s][\color{gray_ulisses}\ttfamily\tiny]{\{-}{-\}},
        morestring=[b]",
        %stringstyle=\color{red},
        showstringspaces=false,
        numberstyle=\tiny,
        numberblanklines=true,
        showspaces=false,
        breaklines=true,
        showtabs=false,
        emph=
        {[1]
                input,output,vhdl,label,if,else,yield,goto,boolean,bits,true,false,function,return,concat
        },
        emphstyle={[1]\textbf}
}

\lstnewenvironment{prehdl}
{\lstset{language=PreHDL,basicstyle=\ttfamily\small,escapechar=\%}}{}

\newcommand{\ra}{\ensuremath{\rightarrow}}
\newcommand{\edit}[1]{\marginpar{\raggedright\tiny\it{#1}}}

\newcommand{\incode}[1]{{\footnotesize{\tt{#1}}}}


\newcommand{\modal}[2]{\ensuremath{\mathsf{#1}_{#2}}}
\newcommand{\kripke}[1]{\ensuremath{\sim_{#1}}}
\renewcommand{\implies}{\ensuremath{\supset}}
\newcommand{\hbind}[1]{\ensuremath{\mbox{{\footnotesize{\texttt{{>}{>}{=}}}}}_{\mbox{\tiny {\it #1}}}}}
\newcommand{\hbindnull}[1]{\ensuremath{\footnotesize{\texttt{{>}{>}}}_{\mbox{\tiny {\it #1}}}}}
\newcommand{\hreturn}[1]{\ensuremath{\mathtt{return}_{\mbox{\tiny {\it #1}}}}}
%\renewcommand{\code}[1]{\ensuremath{\mathtt{#1}}}
\newcommand{\tyjudgment}[3]{\ensuremath{#1 \triangleright #2 : #3 }}
\newcommand{\nil}{\ensuremath{\mathtt{()}}}
\newcommand{\hmodels}{\tiny{\ddtstile{}{}}}
\newcommand*\widebar[1]{%
  \hbox{%
    \vbox{%
      \hrule height 0.5pt % The actual bar
      \kern0.1ex%         % Distance between bar and symbol
      \hbox{%
        \kern-0.1em%      % Shortening on the left side
        \ensuremath{#1}%
        \kern-0.1em%      % Shortening on the right side
      }%
    }%
  }%
} 

\newcommand{\parS}{\ensuremath{{{.}{\mathtt{\&}}{.}}}}
\newcommand{\parA}{\ensuremath{{{.}{\ast}{.}}}}
%\newcommand{\pardev}{\ensuremath{{{\mathtt{<}}{\mathtt{\&}}{\mathtt{>}}}}}
\newcommand{\pardev}{\ensuremath{\mbox{{\texttt{<\&>}}}}}



\usepackage{tthaskell,proof}
\usepackage{beamerthemesplit}
\newcommand{\ttcode}[1]{{\color{red}{\tt{#1}}}}

\newcommand{\forget}[1]{}
\title{CS8440: State Transition Semantics}
\author{Bill Harrison}
\date{\today}

\begin{document}

\frame{\titlepage}

%%%%%%%%%%%%%%%%%%%%%%%%%%%%
%%%%%%%%%%%%%%%%%%%%%%%%%%%%
%%%%%%%%%%%%%%%%%%%%%%%%%%%%
\frame
{
    \frametitle{State Machines \& Security Models}
    
\begin{itemize}
\item Many security models define \emph{``is secure''} in terms of state machines; intuition:
\begin{itemize}
\item partition state space into secure and insecure states
\item $\ldots$ system is secure iff only secure states are reachable from secure states
\end{itemize}

\pause
\item Today: review state machine idea in the form of \emph{transition semantics} for a programming language

\pause
\item Transition Semantics: 
\begin{itemize}
\item Define the meaning of a language with transition rules. Execute input program \<p\> in state \<m_0\>:
\item  \<(p,m_0) \ra (p_1,m_1) \ra \ldots \ra (p_n,m_n) \ra \ldots\>
\item The example we consider is described in: Gunter, ``Semantics of Programming Languages: Structures and Techniques", pages 14-17.
\end{itemize}
\end{itemize}

}

%%%%%%%%%%%%%%%%%%%%%%%%%%%%
%%%%%%%%%%%%%%%%%%%%%%%%%%%%
%%%%%%%%%%%%%%%%%%%%%%%%%%%%
\frame
{
    \frametitle{The Simple Imperative Language with Loops}

\begin{remark}[Abstract Syntax of While]
\begin{haskell}
I &\in&\relax Identifier
\\
%%%%%%%%%%%%%%%%
N &\in&\relax Numeral
\\
%%%%%%%%%%%%%%%%
B &::=&\relax~ \hskwd{true} ~|~ \hskwd{false} ~|~ B\, \hskwd{and}\, B ~|~ B\, \hskwd{or}\, B ~|~ \hskwd{not}\, B ~|~ E\, \hskwd{<}\, E ~|~ E \,\hskwd{=}\, E
\\
%%%%%%%%%%%%%%%%
E &::=&\relax~ N ~|~ I ~|~ E \,+\, E ~|~ E \,*\, E ~|~ E\, -\, E ~|~ - E
\\
%%%%%%%%%%%%%%%%
C &::=&\relax~ \hskwd{skip} ~|~ C\, ; C ~|~ I \hskwd{:=} E ~|~ 
\\
&&\relax~ \hskwd{if}\, B \,\hskwd{then}\, C\, \hskwd{else}\, C \,\hskwd{fi} ~|~ \hskwd{while}\, B\, \hskwd{do}\, C\, \hskwd{od}
\end{haskell}
\end{remark}


}


%%%%%%%%%%%%%%%%%%%%%%%%%%%%
%%%%%%%%%%%%%%%%%%%%%%%%%%%%
%%%%%%%%%%%%%%%%%%%%%%%%%%%%
\frame
{
    \frametitle{The Memory}

\begin{remark}[Memory maps I to \(Z\)]
\begin{haskell}
lookup ~m~ i &=&\relax~ \hsinf{current value of i}
\\
m[i {\mapsto} v] &=&\relax~ \hsinf{new memory s.t. i is bound to v}
\end{haskell}
\end{remark}

\begin{haskell}
lookup ~m[i {\mapsto} v]~j &=&\relax~ 
\left\{
\begin{array}{lcl}
v && i \mbox{ is } j
\\
lookup~ m ~ j && \mbox{otherwise}
\end{array}
\right.
\end{haskell}

}


%%%%%%%%%%%%%%%%%%%%%%%%%%%%
%%%%%%%%%%%%%%%%%%%%%%%%%%%%
%%%%%%%%%%%%%%%%%%%%%%%%%%%%
\frame
{
    \frametitle{E and B semantics}
    
\begin{haskell}
ev_E~ n~ m &=&\relax~ n
\\
ev_E~ i~ m &=&\relax~ lookup~ i ~m
\\
ev_E~ {-}e~ m &=&\relax~ - (ev_E ~e~ m)
\\
\lefteqn{~~~~~\ldots}
\\
ev_B~ \hskwd{true}~ m &=&\relax~ true
\\
ev_B~ (e_1 = e_2)~ m &=&\relax~ (ev_E~e_1~m~ =_Z ~ev_E~e_2~m)
\\
\lefteqn{~~~~~\ldots}
\end{haskell}

\pause
Question: what are the types of \<ev_E\> and \<ev_B\>?

}



%%%%%%%%%%%%%%%%%%%%%%%%%%%%
%%%%%%%%%%%%%%%%%%%%%%%%%%%%
%%%%%%%%%%%%%%%%%%%%%%%%%%%%
\frame
{
    \frametitle{Transition Semantics of While}

\begin{minipage}{1in}
\begin{haskell}
(i := e, m) \ra m[i\mapsto ev_E ~e ~m]
\end{haskell}
\end{minipage}
\begin{minipage}{1in}
\begin{haskell}
(\hskwd{skip},m) \ra m
\end{haskell}
\end{minipage}

\begin{minipage}{1in}
\begin{haskell}
\infer[]{(c_1 ; c_2, m) \ra (c_1' ; c_2,m')}{(c_1,m) \ra (c_1',m')}
\end{haskell}
\end{minipage}
\begin{minipage}{1in}
\begin{haskell}
\infer[]{(c_1 ; c_2, m) \ra (c_2,m')}{(c_1,m) \ra m'}
\end{haskell}
\end{minipage}

\begin{center}
\begin{minipage}{1in}
\begin{haskell}
\infer[]{(\hskwd{if}~ b ~\hskwd{then}~ c_1~ \hskwd{else}~ c_2 ~\hskwd{fi},m) \ra (c_1,m)}{ev_B ~b ~m = true}
\end{haskell}
\end{minipage}
\begin{minipage}{1in}
\begin{haskell}
\infer[]{(\hskwd{if}~ b ~\hskwd{then}~ c_1~ \hskwd{else}~ c_2 ~\hskwd{fi},m) \ra (c_2,m)}{ev_B ~b ~m = false}
\end{haskell}
\end{minipage}
\end{center}

}

%%%%%%%%%%%%%%%%%%%%%%%%%%%%
%%%%%%%%%%%%%%%%%%%%%%%%%%%%
%%%%%%%%%%%%%%%%%%%%%%%%%%%%
\frame
{
    \frametitle{Transition Semantics of While (cont'd)}

\begin{center}
\begin{haskell}
\infer[]{(\hskwd{while}~ b ~\hskwd{do}~ c~ \hskwd{od},m) \ra (c ; \hskwd{while}~ b ~\hskwd{do}~ c~ \hskwd{od},m)}{ev_B ~b ~m = true}
\end{haskell}
\end{center}

\begin{center}
\begin{haskell}
\infer[]{(\hskwd{while}~ b ~\hskwd{do}~ c~ \hskwd{od},m) \ra m}{ev_B ~b ~m = false}
\end{haskell}
\end{center}

}

\frame
{
\frametitle{In Class Exercise}

Formulate the transition semantics for While in Haskell.

\begin{enumerate}

\item Define each of E, B, and C as \<\hskwd{data}\> declarations;

\item Define the Memory data type next;

\item Define \<ev_E\> and \<ev_B\> as Haskell functions;

\item Finally, define the transitions for C as a function of type: \<(C,Memory) \ra Memory\>.

\end{enumerate}

}

\begin{frame}[fragile]
\frametitle{1. Define each of E, B, and C as \<\hskwd{data}\> declarations}

\begin{smallcode}
type Ident  = String
type Number = Int%\pause%

data E = %\pause%N Number
       | I Ident
       | Add E E | Mult E E | Subt E E | Inv E

data B = %\pause%T | F | Conj B B 
       | Disj B B | Negt B | LTC E E | EQL E E

data C = %\pause%Skip
       | Asn Ident E
       | Seq C C
       | IfElse B C C | While B C
\end{smallcode}

\end{frame}


%%%%
%%%%
\begin{frame}[fragile]
\frametitle{2. Define the Memory data type.}

\begin{smallcode}
type Memory = [(Ident, Number)]

--
-- memory look-up
--
lkup :: Memory -> Ident -> Number
lkup ((x, n):ms) x' = if x == x' then n else lkup ms x'
lkup [ ] _          = error "oh snap, you did something bad!"

\end{smallcode}

\end{frame}

%%%%
%%%%
\begin{frame}[fragile]
\frametitle{3. Define \<ev_E\> and \<ev_B\> as Haskell functions}

\begin{smallcode}
--
-- evaluate a Boolean expression:
--
evB :: %\pause%Memory -> B -> Bool%\pause%
evB _  T           = True%\pause%
evB _  F           = False%\pause%
evB m (Conj b1 b2) = (evB m b1) && (evB m b2)%\pause%
evB m (Disj b1 b2) = (evB m b1) || (evB m b2)
evB m (Negt b)     = not (evB m b)
evB m (LTC e1 e2)  = (evE m e1) < (evE m e2)
evB m (EQL e1 e2)  = (evE m e1) == (evE m e2)
--
-- evaluate an arithmetic expression:
--
evE ::%\pause% Memory -> E -> Number%\pause%
evE _ (N n)        = n%\pause%
evE m (I x)        = lkup m x%\pause%
evE m (Add n1 n2)  = (evE m n1) + (evE m n2)%\pause%
evE m (Mult n1 n2) = (evE m n1) * (evE m n2)
evE m (Subt n1 n2) = (evE m n1) - (evE m n2)
evE m (Inv n)      = -(evE m n)
\end{smallcode}

\end{frame}

%%%%
%%%%
\begin{frame}[fragile]
\frametitle{4. Define C transitions as a function of type: \<(C,Memory) \ra Memory\>}

\begin{smallcode}
type Trans  = (C, Memory) -> Memory

exec :: Trans
exec (Skip,m)            = ...
exec (Asn i e,m)        = ...
exec (Seq c1 c2,m)      = ...
exec (IfElse b c1 c2,m) = ...
exec (While b c,m)      = ...
\end{smallcode}

{\tiny
\begin{minipage}{1in}
\begin{haskell}
(i := e, m) \ra m[i\mapsto ev_E ~e ~m]
\end{haskell}
\end{minipage}
\begin{minipage}{1in}
\begin{haskell}
(\hskwd{skip},m) \ra m
\end{haskell}
\end{minipage}

\begin{minipage}{1in}
\begin{haskell}
\infer[]{(c_1 ; c_2, m) \ra (c_1' ; c_2,m')}{(c_1,m) \ra (c_1',m')}
\end{haskell}
\end{minipage}
\begin{minipage}{1in}
\begin{haskell}
\infer[]{(c_1 ; c_2, m) \ra (c_2,m')}{(c_1,m) \ra m'}
\end{haskell}
\end{minipage}
\begin{center}
\begin{minipage}{1in}
\begin{haskell}
\infer[]{(\hskwd{if}~ b ~\hskwd{then}~ c_1~ \hskwd{else}~ c_2 ~\hskwd{fi},m) \ra (c_1,m)}{ev_B ~b ~m = true}
\end{haskell}
\end{minipage}
\begin{minipage}{1in}
\begin{haskell}
\infer[]{(\hskwd{if}~ b ~\hskwd{then}~ c_1~ \hskwd{else}~ c_2 ~\hskwd{fi},m) \ra (c_2,m)}{ev_B ~b ~m = false}
\end{haskell}
\end{minipage}
\end{center}
}
\end{frame}

    
%%%%
%%%%
\begin{frame}[fragile]
\frametitle{4. Define C transitions as a function of type: \<(C,Memory) \ra Memory\>}

\begin{smallcode}
type Trans  = (C, Memory) -> Memory

exec :: Trans
exec (Skip,m)           = m
exec (Asn i e,m)        = (i,evE m e) : m
exec (Seq c1 c2,m)      = exec (c2,exec (c1,m))
exec (IfElse b c1 c2,m) = if evB m b 
                            then exec (c1,m) 
                            else exec (c2,m)
exec (While b c,m)      = if evB m b 
                            then exec (While b c,exec (c,m))
                            else m
\end{smallcode}

\end{frame}
\end{document}












