\documentclass[xcolor=pdftex,dvipsnames,table]{beamer}

\usepackage{listings}
\usepackage{haskell}
\renewcommand{\ttdefault}{pcr}
\definecolor{gray_ulisses}{gray}{0.55}
\definecolor{castanho_ulisses}{rgb}{0.71,0.33,0.14}
\definecolor{preto_ulisses}{rgb}{0.41,0.20,0.04}
\definecolor{green_ulises}{rgb}{0.2,0.75,0}
\lstdefinelanguage{HaskellUlisses} {
        basicstyle=\itfamily\small,
        sensitive=true,
        morecomment=[l][\ttfamily\tiny]{--},
        morecomment=[s][\ttfamily\tiny]{\{-}{-\}},
%       morecomment=[l][\color{gray_ulisses}\ttfamily]{--},
%       morecomment=[s][\color{gray_ulisses}\ttfamily]{\{-}{-\}},
%       morecomment=[l][\color{gray_ulisses}\ttfamily\tiny]{--},
%       morecomment=[s][\color{gray_ulisses}\ttfamily\tiny]{\{-}{-\}},
        morestring=[b]",
        %stringstyle=\color{red},
        showstringspaces=false,
        numberstyle=\tiny,
        numberblanklines=true,
        showspaces=false,
        breaklines=true,
        showtabs=false,
        emph=
        {[1]
                        FilePath,IOError,abs,acos,acosh,all,and,any,appendFile,approxRational,asTypeOf,asin,
                asinh,atan,atan2,atanh,basicIORun,break,catch,ceiling,chr,compare,concat,concatMap,
                const,cos,cosh,curry,cycle,decodeFloat,denominator,digitToInt,div,divMod,drop,
                dropWhile,either,elem,encodeFloat,enumFrom,enumFromThen,enumFromThenTo,enumFromTo,
                error,even,exponent,fail,filter,flip,floatDigits,floatRadix,floatRange,floor,
                fmap,foldl,foldl1,foldr,foldr1,fromDouble,fromEnum,fromInt,fromInteger,fromIntegral,
                fromRational,gcd,getChar,getContents,getLine,head,id,inRange,index,init,intToDigit,
                interact,ioError,isAlpha,isAlphaNum,isAscii,isControl,isDenormalized,isDigit,isHexDigit,
                isIEEE,isInfinite,isLower,isNaN,isNegativeZero,isOctDigit,isPrint,isSpace,isUpper,iterate,iter,iterS,
                last,lcm,length,lex,lexDigits,lexLitChar,lines,log,logBase,lookup,map,mapM,mapM_,max,
                maxBound,maximum,maybe,min,minBound,minimum,mod,negate,not,notElem,null,numerator,odd,
                or,ord,pi,pred,primExitWith,print,product,properFraction,putChar,putStr,putStrLn,quot,
                quotRem,range,rangeSize,read,readDec,readFile,readFloat,readHex,readIO,readInt,readList,readLitChar,
                readLn,readOct,readParen,readSigned,reads,readsPrec,realToFrac,recip,refold,rem,replicate,
                reverse,round,scaleFloat,scanl,scanl1,scanr,scanr1,seq,sequence,sequence_,show,showChar,showInt,
                showList,showLitChar,showParen,showSigned,showString,shows,showsPrec,significand,signum,sin,
                sinh,snd,span,splitAt,sqrt,subtract,succ,sum,tail,take,takeWhile,tan,tanh,threadToIOResult,toEnum,
                toInt,toInteger,toLower,toRational,toUpper,truncate,uncurry,unlines,until,unwords,unzip,
                unzip3,userError,words,writeFile,zip,zip3,zipWith,zipWith3,listArray,doParse,lift,signal,get,put,extrude,return
        },
        emphstyle={[1]\textbf},
        emph=
        {[2]
                Double,Either,IO,Ordering,Rational,Ratio,ReadS,ShowS,
                Word8,InPacket,ReT,StT,I
        },
        emphstyle={[2]\textbf},
        emph=
        {[3]
                case,class,deriving,do,else,if,import,in,infixl,infixr,instance,let,
                module,of,primitive,then,data,type,vhdl,newtype
        },
        emphstyle={[3]\textbf},
        emph=
        {[4]
                quot,rem,div,mod,elem,notElem,seq
        },
%        emphstyle={[4]\color{castanho_ulisses}\textbf},
        emphstyle={[4]\textbf},
        emph=
        {[5]
                False,Just,Left,Nothing,Right,Show,Eq,Ord,Num
        },
%        emphstyle={[5]\color{preto_ulisses}\textbf}
        emphstyle={[5]\textbf}
}

\lstnewenvironment{footcode}
}
{}
%{\hrulesmallskip} not sure what this is.

%\lstnewenvironment{hcode}
%{%\textbf{Haskell Code} \hspace{1cm} \hrulefill
% \lstset{language=HaskellUlisses,    basicstyle=\ttfamily\small,escapechar=\%}}
%{}

\lstnewenvironment{hcode}
}
{}

\lstnewenvironment{smallcode}
{%\hspace{1cm}
 \lstset{language=HaskellUlisses,       basicstyle=\ttfamily\small,escapechar=\%
}}
{}
%{\hrulesmallskip} not sure what this is.

\lstnewenvironment{tinycode}
{%\hspace{1cm}
 \lstset{language=HaskellUlisses,       basicstyle=\ttfamily\scriptsize,escapechar=\%
}}
{}

\lstnewenvironment{teenycode}
{%\hspace{1cm}
 \lstset{language=HaskellUlisses,       basicstyle=\ttfamily\tiny,escapechar=\%
}}
{}


\lstnewenvironment{newcode}{\lstset{language=Haskell,basicstyle=\scriptsize,identifierstyle=\itshape,stringstyle=\itshape,escapechar=\%}}{}

\lstnewenvironment{normalcode}{\lstset{language=Haskell,
                                       basicstyle=\normalsize\ttfamily,
                                       keywordstyle=\bfseries,
                                       %identifierstyle=\itshape,
                                       %stringstyle=\itshape,
                                       escapechar=\%}}{}

\lstnewenvironment{vhdl}
{\lstset{language=VHDL,basicstyle=\ttfamily\small,escapechar=\%}}{}

\lstdefinelanguage{PreHDL} {
        basicstyle=\ttfamily\small,
        sensitive=true,
        morecomment=[l][\ttfamily\tiny]{//},
%        morecomment=[s][\ttfamily\tiny]{\{-}{-\}},
%       morecomment=[l][\color{gray_ulisses}\ttfamily]{--},
%       morecomment=[s][\color{gray_ulisses}\ttfamily]{\{-}{-\}},
%       morecomment=[l][\color{gray_ulisses}\ttfamily\tiny]{--},
%       morecomment=[s][\color{gray_ulisses}\ttfamily\tiny]{\{-}{-\}},
        morestring=[b]",
        %stringstyle=\color{red},
        showstringspaces=false,
        numberstyle=\tiny,
        numberblanklines=true,
        showspaces=false,
        breaklines=true,
        showtabs=false,
        emph=
        {[1]
                input,output,vhdl,label,if,else,yield,goto,boolean,bits,true,false,function,return,concat
        },
        emphstyle={[1]\textbf}
}

\lstnewenvironment{prehdl}
{\lstset{language=PreHDL,basicstyle=\ttfamily\small,escapechar=\%}}{}

\newcommand{\ra}{\ensuremath{\rightarrow}}
\newcommand{\edit}[1]{\marginpar{\raggedright\tiny\it{#1}}}

\newcommand{\incode}[1]{{\footnotesize{\tt{#1}}}}


\newcommand{\modal}[2]{\ensuremath{\mathsf{#1}_{#2}}}
\newcommand{\kripke}[1]{\ensuremath{\sim_{#1}}}
\renewcommand{\implies}{\ensuremath{\supset}}
\newcommand{\hbind}[1]{\ensuremath{\mbox{{\footnotesize{\texttt{{>}{>}{=}}}}}_{\mbox{\tiny {\it #1}}}}}
\newcommand{\hbindnull}[1]{\ensuremath{\footnotesize{\texttt{{>}{>}}}_{\mbox{\tiny {\it #1}}}}}
\newcommand{\hreturn}[1]{\ensuremath{\mathtt{return}_{\mbox{\tiny {\it #1}}}}}
%\renewcommand{\code}[1]{\ensuremath{\mathtt{#1}}}
\newcommand{\tyjudgment}[3]{\ensuremath{#1 \triangleright #2 : #3 }}
\newcommand{\nil}{\ensuremath{\mathtt{()}}}
\newcommand{\hmodels}{\tiny{\ddtstile{}{}}}
\newcommand*\widebar[1]{%
  \hbox{%
    \vbox{%
      \hrule height 0.5pt % The actual bar
      \kern0.1ex%         % Distance between bar and symbol
      \hbox{%
        \kern-0.1em%      % Shortening on the left side
        \ensuremath{#1}%
        \kern-0.1em%      % Shortening on the right side
      }%
    }%
  }%
} 

\newcommand{\parS}{\ensuremath{{{.}{\mathtt{\&}}{.}}}}
\newcommand{\parA}{\ensuremath{{{.}{\ast}{.}}}}
%\newcommand{\pardev}{\ensuremath{{{\mathtt{<}}{\mathtt{\&}}{\mathtt{>}}}}}
\newcommand{\pardev}{\ensuremath{\mbox{{\texttt{<\&>}}}}}


\usetheme{Berlin}
%\usetheme{Boadilla}
%\usetheme{Singapore}
%\usetheme{Berkeley}
\useoutertheme{infolines}
\usecolortheme{lily}
%\usepackage{beamerthemesplit}
\setbeamertemplate{footline}[page number]{}

\setbeamertemplate{footline}[text line]{}
%\usecolortheme{seahorse}
\useinnertheme[shadow]{rounded} 
\setbeamertemplate{navigation symbols}{}
%\usepackage[absolute,overlay]{textpos} 

%\usepackage[all]{xy}
%\usepackage{xmpmulti}

%\usepackage{pdfpages}
\AtBeginSection[]
{
  \begin{frame}<beamer>
    \frametitle{Outline for section \thesection}
    \tableofcontents[currentsection]
  \end{frame}
}


\usepackage{hyperref}
\hypersetup{
    colorlinks=true,
    linkcolor=blue,
    filecolor=magenta,      
    urlcolor=red,
}
 
\urlstyle{same}

\newenvironment{codeblock}[1][.8]{%
\begin{columns}
\begin{column}{#1\linewidth}
\begin{exampleblock}{}}{%
\end{exampleblock}
\end{column}
\end{columns}} 

\newenvironment{execblock}[1][.8]{%
\begin{columns}
\begin{column}{#1\linewidth}
\begin{block}{}}{%
\end{block}
\end{column}
\end{columns}} 

\def\slideskip{\vskip 0.1in}
\def\frameskip{\vskip 0.1in}


%\begin{frame}
%\begin{figure}[t]
%\includegraphics[width=1\textwidth]{example1.png}
%\end{figure}
%\end{frame}

\title[CS4450]{CS4450/7450\\AoPL, Chapter 6: Computational Strategies}
\subtitle{Principles of Programming Languages}
\author[Bill Harrison]{Dr. William Harrison}
\institute{University of Missouri}
%\date{August 31, 2011}

\begin{document}

\frame{\titlepage}

%%%%%%%%%
%%%%%%%%%
%%%%%%%%%
%%%%%%%%%

\begin{frame}[fragile]
\frametitle{Announcements}

\begin{itemize}

\item We're continuing with William Cook's online textbook, \emph{Anatomy of Programming Languages}. It is available
\href{https://www.cs.utexas.edu/~wcook/anatomy/}{here}. We're in Chapter 6.

\end{itemize}

\end{frame}


%%%%%%%%%%%%%%%%%%%%%%%%%%%%%%%%%%%%%
\section{Error Checking}
%%%%%%%%%%%%%%%%%%%%%%%%%%%%%%%%%%%%%
%%%%%%%%%
%%%%%%%%%
%%%%%%%%%
%%%%%%%%%

%%%%%%%%%
%%%%%%%%%
%%%%%%%%%
%%%%%%%%%

%%%%%%%%%%%%%%%%%%%%%%%%%%%%%%%%%%%%%
\section{Mutable State}
%%%%%%%%%%%%%%%%%%%%%%%%%%%%%%%%%%%%%

%%%%%%%%%
%%%%%%%%%
%%%%%%%%%
%%%%%%%%%
\begin{frame}[fragile]
\frametitle{Mutable State}

\begin{block}{Mutable State}
Mutable state means that the state of a program changes or mutates: that a variable can be assigned a new value or a part of a data structure can be modified.
\end{block}

\vfill

Ex: The values of {\tt x} and {\tt i} change over time:
\begin{hcode}
x = 1;
for (i = 2; i <= 5; i = i + 1) {
  x = x * i;
}
\end{hcode}


\end{frame}
%%%%%%%%%
%%%%%%%%%
%%%%%%%%%
%%%%%%%%%

%%%%%%%%%
%%%%%%%%%
%%%%%%%%%
%%%%%%%%%
\begin{frame}[fragile]
\frametitle{Addresses}

\begin{block}{Addresses}
An address identifies a mutable container that stores a single value, but whose contents can change over time. Addresses are sometimes called locations.
\end{block}

\vfill

\pause

\begin{tabular}{l|l}
Operation & Meaning
\\
\hline\hline
\verb+mutable e+ & Creates a mutable cell with initial value given by \verb+e+
\\
\verb+@e+	& Accesses contents stored at address \verb+e+
\\
\verb+a = e+ & Updates contents at address \verb+a+ to be value of expression \verb+e+
\end{tabular}

\end{frame}
%%%%%%%%%
%%%%%%%%%
%%%%%%%%%
%%%%%%%%%

%%%%%%%%%
%%%%%%%%%
%%%%%%%%%
%%%%%%%%%
\begin{frame}[fragile]
\frametitle{Example}

\begin{hcode}
x = mutable 1;
for (i = mutable 2; @i <= 5; i = @i + 1) {
  x = @x * @i;
}
\end{hcode}

\end{frame}
%%%%%%%%%
%%%%%%%%%
%%%%%%%%%
%%%%%%%%%

%%%%%%%%%
%%%%%%%%%
%%%%%%%%%
%%%%%%%%%
\begin{frame}[fragile]
\frametitle{New Values and Expressions}

\begin{hcode}
data Value = IntV  Int
           | BoolV Bool
           | ClosureV String Exp Env
           | AddressV Int        -- new
  deriving (Eq, Show)

type Memory = [Value]
\end{hcode}

\pause

\begin{hcode}
data Exp = ...
         | Mutable   Exp        -- mutable e
         | Access    Exp        -- @a
         | Assign    Exp Exp    -- a = e
\end{hcode}         

\end{frame}
%%%%%%%%%
%%%%%%%%%
%%%%%%%%%
%%%%%%%%%

%%%%%%%%%
%%%%%%%%%
%%%%%%%%%
%%%%%%%%%
\begin{frame}[fragile]
\frametitle{Illustrating Memory Operations}

\begin{tabular}{l|l}
Step&	Memory\\
\hline\hline
start&	[]
\\\pause
\verb+x = mutable 1;+	& [1] \pause
\\
\verb+i = mutable 2;+&	[1,2]\pause
\\
\verb+x = @x * @i;+	&[2,2]\pause
\\
\verb|i = @i + 1;|	& [2,3]\pause
\\
\verb+x = @x * @i;+&	[6,3]
\\
\verb|i = @i + 1;|&	[6,4]
\\
\verb+x = @x * @i;+&	[24,4]
\\
\verb|i = @i + 1;|	&[24,5]
\\
\verb+x = @x * @i;+&	[120,5]
\\
\verb|i = @i + 1;|	&[120,6]
\end{tabular}

\end{frame}
%%%%%%%%%
%%%%%%%%%
%%%%%%%%%
%%%%%%%%%

\subsection{Pure Functional Operations on Memory}
%%%%%%%%%
%%%%%%%%%
%%%%%%%%%
%%%%%%%%%
\begin{frame}[fragile]
\frametitle{Operations on Memory}

\noindent Accessing Memory:
\begin{hcode}
access i mem = mem !! i
\end{hcode}

\vfill\pause

\noindent Updating Memory:
\begin{hcode}
update :: Int -> Value -> Memory -> Memory
update addr val mem =
  let (before, _ : after) = splitAt addr mem in
    before ++ [val] ++ after
\end{hcode}
\vfill

\pause
\begin{hcode}
ghci> :t splitAt
   splitAt :: Int -> [a] -> ([a], [a])
ghci> splitAt 3 "abcdefg"
   ("abc","defg")
ghci> splitAt 0 "abcdefg"
   ("","abcdefg")
\end{hcode}

\end{frame}
%%%%%%%%%
%%%%%%%%%
%%%%%%%%%
%%%%%%%%%

%%%%%%%%%
%%%%%%%%%
%%%%%%%%%
%%%%%%%%%
\begin{frame}[fragile]
\frametitle{Whither {\texttt{evaluate}}?}

\begin{itemize}
\item Currently:
\begin{hcode}
    evaluate :: Exp -> Env -> Value
\end{hcode}

\pause
\item Defn. A {\bf stateful computation} is a function that takes an input state and returns a value and an output state.


\pause
\item Now a stateful computation:
\begin{hcode}
    evaluate :: Exp -> Env -> %\textcolor{red}{Memory -> (Value,Memory)}%
\end{hcode}


\pause
\item Generalizing:
\begin{hcode}
    type Stateful a = Memory -> (a,Memory)
    evaluate :: Exp -> Env -> Memory -> Stateful Value
\end{hcode}


\end{itemize}

\end{frame}
%%%%%%%%%
%%%%%%%%%
%%%%%%%%%
%%%%%%%%%

%%%%%%%%%
%%%%%%%%%
%%%%%%%%%
%%%%%%%%%
\begin{frame}[fragile]
\frametitle{Evaluation Rules}

\begin{hcode}
evaluate (Mutable e) env mem = 
   let
      (ev, mem') = evaluate e env mem
   in 
      (AddressV (length mem'), mem' ++ [ev])
\end{hcode}

\pause

\begin{hcode}
evaluate (Access a) env mem =
  let 
     (AddressV i, mem') = evaluate a env mem 
  in
     (access i mem', mem')
\end{hcode}


\end{frame}
%%%%%%%%%
%%%%%%%%%
%%%%%%%%%
%%%%%%%%%

%%%%%%%%%
%%%%%%%%%
%%%%%%%%%
%%%%%%%%%
\begin{frame}[fragile]
\frametitle{Evaluation Rules (cont'd)}

\begin{hcode}
evaluate (Assign a e) env mem =
  let 
     (AddressV i, mem') = evaluate a env mem
  in  
     let 
        (ev, mem'') = evaluate e env mem' 
     in
        (ev, update i ev mem'')
\end{hcode}

\begin{hcode}
evaluate (Binary op a b) env mem =
  let 
     (av, mem') = evaluate a env mem 
  in
     let 
        (bv, mem'') = evaluate b env mem' 
     in
        (binary op av bv, mem'')
\end{hcode}

\end{frame}
%%%%%%%%%
%%%%%%%%%
%%%%%%%%%
%%%%%%%%%

%%%%%%%%%
%%%%%%%%%
%%%%%%%%%
%%%%%%%%%
\begin{frame}[fragile]
\frametitle{A pattern is forming...}

\begin{itemize}
\item Recall:
\begin{hcode}
evaluate (Mutable e) env mem = 
   let (ev, mem') = evaluate e env mem
   in  (AddressV (length mem'), mem' ++ [ev])
\end{hcode}

\pause

\item The pattern:
\begin{hcode}
 \ mem -> 
   let
      (val, mem') = %\textcolor{red}{{\it first-part}}% mem
   in 
      %\textcolor{red}{{\it next-part}}% val mem'
\end{hcode}

\pause

\item Generalizing as a higher-order function:
\begin{hcode}
(>>=) :: Stateful a -> (a -> Stateful b) -> Stateful b
%\textcolor{red}{{\it first-part}}% >>= %\textcolor{red}{{\it next-part}}% = 
 \ mem ->
   let
      (val, mem') = %\textcolor{red}{{\it first-part}}% mem
   in 
      %\textcolor{red}{{\it next-part}}% val mem'
\end{hcode}
\end{itemize}

\end{frame}
%%%%%%%%%
%%%%%%%%%
%%%%%%%%%
%%%%%%%%%



\end{document}