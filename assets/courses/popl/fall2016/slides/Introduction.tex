\documentclass{beamer}

%\usetheme{Malmoe}
\usetheme{Hannover}
\usecolortheme{dolphin}

\newtheorem{remark}{}

\usepackage{listings}
\usepackage{haskell,proof}
%\usepackage{beamerthemesplit}
\usepackage{fancyvrb,alltt}

\lstnewenvironment{newcode}{\lstset{language=Haskell,basicstyle=\scriptsize,escapechar=\@}}{}
\newcommand{\ttcode}[1]{{\color{red}{\tt{#1}}}}
\newcommand{\ra}{\rightarrow}

\newcommand{\forget}[1]{}
\title{CS4450/7450: Introduction}
\author{Professor William L. Harrison}
\date{\today}

\begin{document}

\section{Administrivia}

\frame{\titlepage}


%%%%%%%%%%%%%%%%%%%%%%%%%%%%
%%%%%%%%%%%%%%%%%%%%%%%%%%%%
%%%%%%%%%%%%%%%%%%%%%%%%%%%%
\frame
{
    \frametitle{CS4450/7450: Principles of Programming Languages}
    
\begin{itemize}
\item What is this course about?
\begin{itemize}
\item Programming language Design
\pause
\item Data abstraction, Language Classes, Types
\end{itemize}
\pause

\item Administrivia
\begin{itemize}
\item Grading
\item Texts
\end{itemize}
\pause

\item First Homework is out. 
\begin{itemize}
\item See course webpage (below)
\end{itemize}

\end{itemize}
}

%%%%%%%%%%%%%%%%%%%%%%%%%%%%
%%%%%%%%%%%%%%%%%%%%%%%%%%%%
%%%%%%%%%%%%%%%%%%%%%%%%%%%%
\frame
{
    \frametitle{Grading}
    
\begin{itemize}
\item Midterm (9/21): 25\%

\item Written Homeworks + Programming: 40\%

\item Final: 35\%

\end{itemize}

\begin{remark}[Regrade Policy]
Requests for re-grades must be made in writing within 7 days of 
receiving graded HW/test. There will be absolutely no exceptions.
\end{remark}
}

\begin{frame}[fragile]

Professor,

While you were away, I attempted to inquire about my grade via email and your auto email responded saying you were out of the country. I was inquiring about my grade because as a Senior slated to graduate in the fall {\color{blue}{I was hoping to earn an A- in your class after we had taken the final. I needed the A- for gpa purposes for job applications}} that I had to begun to fill out for when I graduate in the fall. {\color{brown}{I work in Dr. Kazic's Lab and she always speaks highly of you}} so I was hopeful that there may have either been a miscalculation on my final or class attendance points or a way for me to earn a few more points so that I may raise my grade and graduate in the fall.

{\color{red}{Is there any way for me to raise my grade a degree?}}

Gratitude,
    ...
\end{frame}

\begin{frame}[fragile]

{\small
%\begin{Verbatim}[commandchars=\\\{\}]
\begin{alltt}
I have more than three days left.  
The final isn't till next Thursday. 
I'm not asking for pity, I'm asking 
that you do help me with specific material 
and not throw me to your TA's.  I talked 
to you about this before Thanksgiving, but 
was out of town all last week for an interview 
with Amazon.  I wasn't able to come in then 
so there was no point in talking to you about 
it.  I've really been trying to keep up, but 
I just haven't been able to.  

I'll come in multiple times, and only ask 
specific questions.  I'm not just going to 
come in and say "teach me".  
\end{alltt}}

\end{frame}

\begin{frame}[fragile]

{\small
\begin{alltt}
Quite frankly, I'm disappointed that this is 
the response I get when I ask for help.  You 
still haven't given me potential times to come 
in, and I assume that by including your TA's I'm 
going to get the typical response of "go talk to 
your TA's".  I don't pay this University though 
to talk to TA's.  I pay to get help from a professor. 

Also, you are the third professor that has been stubborn 
to work with me with traveling this semester. I'm not 
taking vacations, I'm trying to get a job at some of 
the biggest tech companies in the US (Google, Amazon, 
and Microsoft).  This University is supposed to enable 
me to succeed, not impede me.  

So, I'll ask again. When can I come talk to you about 
the material in class so I can pass this class, and 
more importantly, understand the material?  
\end{alltt}
%%\end{Verbatim}
}

\end{frame}

%%%%%%%%%%%%%%%%%%%%%%%%%%%%
%%%%%%%%%%%%%%%%%%%%%%%%%%%%
%%%%%%%%%%%%%%%%%%%%%%%%%%%%
\frame
{
    \frametitle{Personnel}
    
\begin{itemize}
\item Instructor: Professor William L. Harrison\\
%\item 
Office: 318 EBN
\pause


\item Teaching Assistant: Chris Hathhorn\\
%\item 
Office: 317 EBN
\end{itemize}
}
%%%%%%%%%%%%%%%%%%%%%%%%%%%%
%%%%%%%%%%%%%%%%%%%%%%%%%%%%
%%%%%%%%%%%%%%%%%%%%%%%%%%%%
\begin{frame}%{fragile}
    \frametitle{On the Web...}
    
\begin{itemize}
\item Blackboard...
\item All announcements, homeworks, syllabus...
%\end{itemize}

\pause
\item A Github repository TBD...
\begin{itemize}

\item Slides, class notes,...
\end{itemize}
\end{itemize}

\end{frame}

%%%%%%%%%%%%%%%%%%%%%%%%%%%%
%%%%%%%%%%%%%%%%%%%%%%%%%%%%
%%%%%%%%%%%%%%%%%%%%%%%%%%%%
\begin{frame}%{fragile}
    \frametitle{Textbooks}

{\bf Required:} \emph{Thinking Functionally with Haskell}, Richard Bird, Cambridge University Press.

{\bf Recommended:} The following are good but not required.    
\begin{itemize}
\item  \emph{Programming in Haskell} by Graham Hutton\\

\item \emph{Learn You a Haskell for Good} by Miran Lipovaca\\
Highly amusing and informative; available online.

\item Course notes by me \\
I will post these occasionally for certain subjects.

\item Slides

\end{itemize}

\end{frame}

%%%%%%%%%%%%%%%%%%%%%%%%%%%%
%%%%%%%%%%%%%%%%%%%%%%%%%%%%
%%%%%%%%%%%%%%%%%%%%%%%%%%%%
\begin{frame}%{fragile}
    \frametitle{Academic Honesty}
    
\begin{itemize}
\item Your work must be your own
\begin{itemize}
\item Discussion is fine and encouraged.
\pause
\item ``Discussion'' means talking with your mouth or chalkboard only.
\item Realize that if you email someone code as an explanation, they may cut and paste it into something they turn in.
\end{itemize}

\pause

\item Consequences of Academic Dishonesty
\begin{itemize}
\item 1st offense: 0 on offending assignment \& warning
\item 2nd offense: Automatic F grade; case forwarded to Provost.
\pause
\item No exceptions!
\end{itemize}

\pause

\item Your continued enrollment in this class implies your consent to these rules.
\end{itemize}

\end{frame}


%%%%%%%%%%%%%%%%%%%%%%%%%%%%
%%%%%%%%%%%%%%%%%%%%%%%%%%%%
%%%%%%%%%%%%%%%%%%%%%%%%%%%%
\begin{frame}[fragile]
    \frametitle{Question: What does this program do? }
    


\begin{verbatim}
              n := i; 
              a := 1; 
              while (n > 0) { 
                   a := a * n; 
                   n := n - 1; 
              }
\end{verbatim}

\end{frame}

%%%%%%%%%%%%%%%%%%%%%%%%%%%%
%%%%%%%%%%%%%%%%%%%%%%%%%%%%
%%%%%%%%%%%%%%%%%%%%%%%%%%%%
\begin{frame}[fragile]
    \frametitle{Functions in Mathematics}
    

\begin{displaymath}
n! = \left\{
     \begin{array}{lcl}
     1 &\mbox{if} & n=0
\\
     n * (n-1)!&\mbox{if} & n > 0
     \end{array}\right.
\end{displaymath}
\pause

What does this have to do with that? 
\begin{verbatim}
              n := i; 
              a := 1; 
              while (n > 0) { 
                   a := a * n; 
                   n := n - 1; 
              }
\end{verbatim}

\end{frame}

%%%%%%%%%%%%%%%%%%%%%%%%%%%%
%%%%%%%%%%%%%%%%%%%%%%%%%%%%
%%%%%%%%%%%%%%%%%%%%%%%%%%%%
\begin{frame}[fragile]
    \frametitle{Your First Haskell Function}
    

\begin{displaymath}
n! = \left\{
     \begin{array}{lcl}
     1 &\mbox{if} & n=0
\\
     n * (n-1)!&\mbox{if} & n > 0
     \end{array}\right.
\end{displaymath}
\pause

What does this have to do with that? 
\begin{verbatim}
      fac n = if n==0 then 1 else n * (fac (n-1))
\end{verbatim}

\end{frame}


\section{Sets}

%%%%%%%%%%%%%%%%%%%%%%%%%%%%
%%%%%%%%%%%%%%%%%%%%%%%%%%%%
%%%%%%%%%%%%%%%%%%%%%%%%%%%%
\frame
{
    \frametitle{Preliminaries}
    

A Quick Review of the Sets: 
\begin{itemize}
\item Relations on sets:  $\in ~ \subseteq$
\item Operations on sets: $\cup~ \cap~ \times ~ +$ 
\item Functions as sets
\item Relations as sets
\end{itemize}

}

\subsection{Relations}
%
%%%%%%%%%%%%%%%%%%%%%%%%%%%%%
%%%%%%%%%%%%%%%%%%%%%%%%%%%%%
%%%%%%%%%%%%%%%%%%%%%%%%%%%%%
\frame
{
    \frametitle{Relations on sets: $\in ~ \subseteq$}

\begin{itemize}    
\item Membership: 
\(
x \in S
\)
\\
\[ 
\begin{array}{lcl}
1 \in \{1,2,3\}
\\
\pause
4 \not\in \{1,2,3\}
\end{array}
\]
\pause
\item Empty set: written $\emptyset$ or $\{\}$. For any object $x$, $x \not\in \{\}$.

\pause

\item Extensionality: two sets are equal if, and only if, they have the same elements. For sets $S$ and $T$, $S = T$ is defined by:
\[
\mbox{for all }x, ~x\in S \mbox{ iff } x \in T
\]
\end{itemize}
}

\frame{
\frametitle{What do you think \& why?}
\begin{remark}[True or False:]
\pause
\begin{enumerate}
\item $\emptyset = \{ 0 \}$
\pause
\item $x \in \{x\}$
\pause
\item $\emptyset = \{\emptyset \}$
\pause
\item $\emptyset \in \{ \emptyset \}$
\pause
\item $\{0,\emptyset\} = \{ 0 \}$
\end{enumerate}
\end{remark}
}

%%%%%%%%%%%%%%%%%%%%%%%%%%%%%
%%%%%%%%%%%%%%%%%%%%%%%%%%%%%
%%%%%%%%%%%%%%%%%%%%%%%%%%%%%
\frame
{
    \frametitle{Relations on sets: $\subseteq$}

\begin{itemize}    
\item Subset: $S \subseteq T$ means  every element of $S$ is an element of $T$: 
\[
\mbox{for all } x, \mbox{ if } x \in S, \mbox{ then } x \in T
\]
\pause
\\
\item Examples
\[ 
\begin{array}{lcl}
\{1\} \subseteq \{1,2,3\}
\\
\pause
\{1,2,3\} \subseteq \{1,2,3\}
\\
\pause
\{1,2,3,4\} \not\subseteq \{1,2,3\}
\\
\pause
\{1,2,3,4\} \not\subseteq \{\}
\\
\pause
\{\} \subseteq \{1,2,3\}
\\
\pause
\{\} \subseteq \{\}
\end{array}
\]
\pause
\item Proper subset  $S \subset T$: means  $S \subseteq T$ and $S \not= T$.

\pause

\item Q: if $S \subseteq T$ and $T \subseteq S$, then $S = T$?

\end{itemize}

}

\subsection{Ordered Pairs}
%%%%%%%%%%%%%%%%%%%%%%%%%%%%%
%%%%%%%%%%%%%%%%%%%%%%%%%%%%%
%%%%%%%%%%%%%%%%%%%%%%%%%%%%%
\frame{
\frametitle{Ordered pairs}

\begin{remark}[Ordered Pair]
\begin{itemize}
\item An {\bf{order pair}} of two objects $x$ and $y$ is written $(x,y)$. 
\item $x$ and $y$ are the {\bf{first and second components}}, resp.
\item $(x,y) = (u,v)$ iff $x=u$ and $y = v$.
\end{itemize}
\end{remark}

}

\section{Set Operations}

%%%%%%%%%%%%%%%%%%%%%%%%%%%%%
%%%%%%%%%%%%%%%%%%%%%%%%%%%%%
%%%%%%%%%%%%%%%%%%%%%%%%%%%%%
\frame
{
    \frametitle{Operations on sets: $\cup~ \cap~ \times$}

    
\begin{itemize}
\item Union of sets: $A \cup B$. For two sets, $A$ and $B$, 
\[
x \in (A \cup B) \mbox{ iff } x \in A \mbox{ or } x \in B
\]
\pause
\item Intersection of sets: $A \cap B$. For two sets, $A$ and $B$, 
\[
x \in (A \cap B) \mbox{ iff } x \in A \mbox{ and } x \in B
\]
\pause
\item Product of sets: $A \times B$. For two sets, $A$ and $B$, 
\[
(x,y) \in (A \times B) \mbox{ iff } x \in A \mbox{ and } y \in B
\]
\pause
\end{itemize}

}

%%%%%%%%%%%%%%%%%%%%%%%%%%%%%
%%%%%%%%%%%%%%%%%%%%%%%%%%%%%
%%%%%%%%%%%%%%%%%%%%%%%%%%%%%
\frame
{
    \frametitle{Representing Relations as Sets}
    
\begin{itemize}
\item A typical relation is less-than on integers $Z = \{\ldots,-1,0,1,\ldots\}$; e.g., $x < y$.
\pause

\item We can represent such relations as particular sets. 
\[
R_< = \{ (0,1), (-99,1001), \ldots \}
\]
In fact, any two-place relation on integers is a subset of $Z \times Z$.
\pause

\item For $i,j \in Z$ and relation $R \subseteq Z \times Z$, 
\[
 R \,i \,j ~\mbox{holds} \mbox{ iff } (i,j) \in R
\]
\end{itemize}

}

%%%%%%%%%%%%%%%%%%%%%%%%%%%%%
%%%%%%%%%%%%%%%%%%%%%%%%%%%%%
%%%%%%%%%%%%%%%%%%%%%%%%%%%%%
\frame
{
    \frametitle{Functions as Sets}
    
\begin{itemize}
\item The same treatment holds for functions
\pause

\item Factorial on $N = \{0,1,\ldots\}$ can be represented as the following subset of $N \times N$:
\[
fac = \{ (0,1), (1,1), (2,2), (3,6), (4,24), \ldots \}
\]
\pause

\item We say that $fac~ n = m$ iff $(n,m) \in fac$
\pause

\item Function Property: a relation $R \subseteq N \times N$ is a function, iff
\[
\mbox{for all }n \in N, \mbox{ if } (n,p) \in R \mbox{ and } (n,q) \in R \mbox{ then } p=q
\]
\pause 

\item Function $f \subseteq N \times N$ is {\bf total} iff for all $n\in N$, there is an $m\in N$ such that $(n,m) \in f$

\end{itemize}


}


\end{document}
    












