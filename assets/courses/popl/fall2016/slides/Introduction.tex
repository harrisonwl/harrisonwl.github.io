\documentclass{beamer}

%\usetheme{Malmoe}
\usetheme{Hannover}
\usecolortheme{dolphin}

\newtheorem{remark}{}

\usepackage{listings}
\usepackage{haskell,proof}
%\usepackage{beamerthemesplit}
\usepackage{fancyvrb,alltt}

\lstnewenvironment{newcode}{\lstset{language=Haskell,basicstyle=\scriptsize,escapechar=\@}}{}
\newcommand{\ttcode}[1]{{\color{red}{\tt{#1}}}}
\newcommand{\ra}{\rightarrow}
\setbeamertemplate{navigation symbols}{}

\newcommand{\forget}[1]{}
\title{CS4450/7450: Introduction}
\author{Professor William L. Harrison}
\date{\today}

\begin{document}

\section{Administrivia}

\frame{\titlepage}


%%%%%%%%%%%%%%%%%%%%%%%%%%%%
%%%%%%%%%%%%%%%%%%%%%%%%%%%%
%%%%%%%%%%%%%%%%%%%%%%%%%%%%
\frame
{
    \frametitle{CS4450/7450: Principles of Programming Languages}
    
\begin{itemize}
\item What is this course about?
\begin{itemize}
\item Programming language Design
\pause
\item Data abstraction, Language Classes, Types
\item Explore these ideas by writing interpreters in Haskell
\end{itemize}
\pause

\item Administrivia
\begin{itemize}
\item Grading
\item Texts
\end{itemize}
\pause

\item Course Webpage
\begin{itemize}
\item \url{https://harrisonwl.github.io/doc/cs4450.html}
\item All announcements and scores will be handled by blackboard site (not created yet).
\end{itemize}

\end{itemize}
}

%%%%%%%%%%%%%%%%%%%%%%%%%%%%
%%%%%%%%%%%%%%%%%%%%%%%%%%%%
%%%%%%%%%%%%%%%%%%%%%%%%%%%%
\frame
{
    \frametitle{Grading}
    
\begin{itemize}
\item Midterm (9/28): 25\%

\item Written Homeworks + Programming: 40\%

\item Final: 35\%

\end{itemize}

\begin{remark}[Regrade Policy]
Requests for re-grades must be made in writing within 7 days of 
receiving graded HW/test. There will be absolutely no exceptions.
\end{remark}
}



%%%%%%%%%%%%%%%%%%%%%%%%%%%%
%%%%%%%%%%%%%%%%%%%%%%%%%%%%
%%%%%%%%%%%%%%%%%%%%%%%%%%%%
\frame
{
    \frametitle{Personnel}
    
\begin{itemize}
\item Instructor: Professor William L. Harrison\\
%\item 
Office: 318 EBN\\
Office hours: \emph{By appointment; email me first}\\



\item Teaching Assistant: \emph{TBD}
\end{itemize}
}

%%%%%%%%%%%%%%%%%%%%%%%%%%%%
%%%%%%%%%%%%%%%%%%%%%%%%%%%%
%%%%%%%%%%%%%%%%%%%%%%%%%%%%
\begin{frame}%{fragile}
    \frametitle{Textbooks}
    
Both texts are online:    
\begin{itemize}
\item \emph{Learn You a Haskell for Good} by Miran Lipovaca\\
Highly amusing and informative; available online.

\item \emph{Anatomy of Programming Languages} by William Cook.


\item Slides, of course.

\end{itemize}

The following are good but not required.    
\begin{itemize}
\item  \emph{Programming in Haskell} by Graham Hutton\\

\item \emph{Thinking Functionally with Haskell}, Richard Bird, Cambridge University Press.

\end{itemize}

\end{frame}



\begin{frame}[fragile]
\frametitle{What is this course about?}

Why would you study interpreters and compilers? 
\pause
\begin{itemize}
\item Writing interpreters/compilers involves a lot of technical skills that you need to use together. 
\begin{itemize}
\item Writing interpreter/compiler helps improve those skills to become a better software developer in general. 
\end{itemize}
\pause

\item You really want to know how computers work. 
\begin{itemize}

\item Often interpreters and compilers look like magic. And you shouldn't be comfortable with that magic. 

\item You want to demystify the process of building an interpreter and a compiler, understand how they work, and get in control of things.
\end{itemize}
\pause

\item You want to create your own programming language or domain specific language. 
\begin{itemize}
\item If you create one, you will also need to create either an interpreter or a compiler for it. 
\item Recently, there has been a resurgence of interest in new  languages: Elixir, Go, Rust just to name a few.
\end{itemize}
\end{itemize}

\end{frame}
%%%%%%%%%%%%%%%%%%%%%%%%%%%%
%%%%%%%%%%%%%%%%%%%%%%%%%%%%
%%%%%%%%%%%%%%%%%%%%%%%%%%%%
\begin{frame}%{fragile}
    \frametitle{Academic Honesty}
    
\begin{itemize}
\item Your work must be your own
\begin{itemize}
\item Discussion is fine and encouraged.
\pause
\item ``Discussion'' means talking with your mouth or chalkboard only.
\item Realize that if you email someone code as an explanation, they may cut and paste it into something they turn in.
\end{itemize}

\pause

\item Consequences of Academic Dishonesty
\begin{itemize}
\item 1st offense: 0 on offending assignment \& warning
\item 2nd offense: Automatic F grade; case forwarded to Provost.
\pause
\item No exceptions!
\end{itemize}

\pause

\item Your continued enrollment in this class implies your consent to these rules.
\end{itemize}

\end{frame}

%%%%%%%%%%%%%%%%%%%%%%%%%%%%
%%%%%%%%%%%%%%%%%%%%%%%%%%%%
%%%%%%%%%%%%%%%%%%%%%%%%%%%%
\begin{frame}[fragile]
    \frametitle{Email Use \& Abuse}

\begin{itemize}

\item I will only answer emails from 9am-5pm Monday-Friday  
\begin{itemize}
\item ...actual emergencies excepted. 
\end{itemize}
\pause

\item It is completely at my discretion whether I answer your email.

\pause
\item Rude, snarky, obnoxious, and/or threatening emails will not be answered and, if appropriate, will be forwarded on to my department chairman and/or provost.

\end{itemize}
    
\end{frame}

\section{Advice}

%%%%%%%%%%%%%%%%%%%%%%%%%%%%
%%%%%%%%%%%%%%%%%%%%%%%%%%%%
%%%%%%%%%%%%%%%%%%%%%%%%%%%%
\begin{frame}[fragile]
    \frametitle{How to get an A in CS4450/7450}

\begin{itemize}

\item When assigned reading, read it carefully.
\begin{itemize}
\item Don't skim, \emph{read}!
\end{itemize}
\pause

\item When given a programming assignment, start early.
\begin{itemize}
\item Last minute, start-the-night-before may work in other classes, but it's a prescription for disaster in CS4450.
\end{itemize}
\pause 

\item Realize that you alone are responsible for your performance; be disciplined.

\end{itemize}
    
\end{frame}


\begin{frame}[fragile]
\frametitle{How not to get an A in CS4450}
\framesubtitle{Case Study 1}

Professor,

While you were away, I attempted to inquire about my grade via email and your auto email responded saying you were out of the country. I was inquiring about my grade because as a Senior slated to graduate in the fall {\color{blue}{I was hoping to earn an A- in your class after we had taken the final. I needed the A- for gpa purposes for job applications}} that I had to begun to fill out for when I graduate in the fall. {\color{brown}{I work in Dr. Kazic's Lab and she always speaks highly of you}} so I was hopeful that there may have either been a miscalculation on my final or class attendance points or a way for me to earn a few more points so that I may raise my grade and graduate in the fall.

{\color{red}{Is there any way for me to raise my grade a degree?}}

Gratitude,
    ...
\end{frame}

\begin{frame}[fragile]
\frametitle{How not to get an A in CS4450}
\framesubtitle{Case Study 2}

{\small
%\begin{Verbatim}[commandchars=\\\{\}]
\begin{alltt}
{\color{red}{I have more than three days left.  
The final isn't till next Thursday. }}
I'm not asking for pity, I'm asking 
that you do help me with specific material 
and not throw me to your TA's.  I talked 
to you about this before Thanksgiving, but 
was out of town all last week for an interview 
with Amazon.  I wasn't able to come in then 
so there was no point in talking to you about 
it.  I've really been trying to keep up, but 
I just haven't been able to.  

I'll come in multiple times, and only ask 
specific questions.  I'm not just going to 
come in and say "teach me".  
\end{alltt}}

\end{frame}

\begin{frame}[fragile]
\frametitle{How not to get an A in CS4450}
\framesubtitle{Case Study 2, cont'd}

{\small
\begin{alltt}
Quite frankly, I'm disappointed that this is 
the response I get when I ask for help.  You 
still haven't given me potential times to come 
in, and I assume that by including your TA's I'm 
going to get the typical response of "go talk to 
your TA's".  {\color{red}{I don't pay this University though 
to talk to TA's.  I pay to get help from a professor. }}

Also, you are the third professor that has been stubborn 
to work with me with traveling this semester. I'm not 
taking vacations, I'm trying to get a job at some of 
the biggest tech companies in the US (Google, Amazon, 
and Microsoft).  {\color{blue}{This University is supposed to enable 
me to succeed, not impede me.  }}

So, I'll ask again. {\color{green}{When can I come talk to you about 
the material in class so I can pass this class, and 
more importantly, understand the material?}}  
\end{alltt}
%%\end{Verbatim}
}

\end{frame}

\section{Obligatory Memes}

\begin{frame}[fragile]
\includegraphics[scale=0.5]{4450memes/helloworld.jpg}
\pause

Actually, it's just:
\begin{verbatim}
            main = print "Hello World"
\end{verbatim}
\end{frame}

\begin{frame}[fragile]
\begin{center}
\includegraphics[scale=0.65]{4450memes/matrix.jpg}
\end{center}
\end{frame}

\begin{frame}[fragile]
\begin{center}
\includegraphics[scale=0.65]{4450memes/kanye.jpg}
\end{center}
\end{frame}

\begin{frame}[fragile]
\begin{center}
\includegraphics[scale=0.35]{4450memes/purity.png}
\end{center}
\end{frame}

\begin{frame}[fragile]
\begin{center}
\includegraphics[scale=0.45]{4450memes/meme-functions.jpg}
\end{center}
\end{frame}

\begin{frame}[fragile]
\begin{center}
\includegraphics[scale=0.45]{4450memes/monads.png}
\end{center}
\end{frame}

\section{First Example}
%%%%%%%%%%%%%%%%%%%%%%%%%%%%
%%%%%%%%%%%%%%%%%%%%%%%%%%%%
%%%%%%%%%%%%%%%%%%%%%%%%%%%%
\begin{frame}[fragile]
    \frametitle{Question: What does this program do? }
    


\begin{verbatim}
              n := i; 
              a := 1; 
              while (n > 0) { 
                   a := a * n; 
                   n := n - 1; 
              }
\end{verbatim}

\end{frame}

%%%%%%%%%%%%%%%%%%%%%%%%%%%%
%%%%%%%%%%%%%%%%%%%%%%%%%%%%
%%%%%%%%%%%%%%%%%%%%%%%%%%%%
\begin{frame}[fragile]
    \frametitle{Functions in Mathematics}
    

\begin{displaymath}
n! = \left\{
     \begin{array}{lcl}
     1 &\mbox{if} & n=0
\\
     n * (n-1)!&\mbox{if} & n > 0
     \end{array}\right.
\end{displaymath}
\pause

What does this have to do with that? 
\begin{verbatim}
              n := i; 
              a := 1; 
              while (n > 0) { 
                   a := a * n; 
                   n := n - 1; 
              }
\end{verbatim}

\end{frame}

%%%%%%%%%%%%%%%%%%%%%%%%%%%%
%%%%%%%%%%%%%%%%%%%%%%%%%%%%
%%%%%%%%%%%%%%%%%%%%%%%%%%%%
\begin{frame}[fragile]
    \frametitle{Your First Haskell Function}
    

\begin{displaymath}
n! = \left\{
     \begin{array}{lcl}
     1 &\mbox{if} & n=0
\\
     n * (n-1)!&\mbox{if} & n > 0
     \end{array}\right.
\end{displaymath}
\pause

What does this have to do with that? 
\begin{verbatim}
      fac n = if n==0 then 1 else n * (fac (n-1))
\end{verbatim}

\end{frame}


\end{document}
    












