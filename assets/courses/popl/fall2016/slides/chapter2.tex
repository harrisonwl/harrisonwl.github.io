\documentclass{beamer}

\usepackage{listings,color}

\renewcommand{\ttdefault}{pcr}
\definecolor{gray_ulisses}{gray}{0.55}
\definecolor{castanho_ulisses}{rgb}{0.71,0.33,0.14}
\definecolor{preto_ulisses}{rgb}{0.41,0.20,0.04}
\definecolor{green_ulises}{rgb}{0.2,0.75,0}
\lstdefinelanguage{HaskellUlisses} {
        basicstyle=\itfamily\small,
        sensitive=true,
        morecomment=[l][\ttfamily\tiny]{--},
        morecomment=[s][\ttfamily\tiny]{\{-}{-\}},
%       morecomment=[l][\color{gray_ulisses}\ttfamily]{--},
%       morecomment=[s][\color{gray_ulisses}\ttfamily]{\{-}{-\}},
%       morecomment=[l][\color{gray_ulisses}\ttfamily\tiny]{--},
%       morecomment=[s][\color{gray_ulisses}\ttfamily\tiny]{\{-}{-\}},
        morestring=[b]",
        %stringstyle=\color{red},
        showstringspaces=false,
        numberstyle=\tiny,
        numberblanklines=true,
        showspaces=false,
        breaklines=true,
        showtabs=false,
        emph=
        {[1]
                        FilePath,IOError,abs,acos,acosh,all,and,any,appendFile,approxRational,asTypeOf,asin,
                asinh,atan,atan2,atanh,basicIORun,break,catch,ceiling,chr,compare,concat,concatMap,
                const,cos,cosh,curry,cycle,decodeFloat,denominator,digitToInt,div,divMod,drop,
                dropWhile,either,elem,encodeFloat,enumFrom,enumFromThen,enumFromThenTo,enumFromTo,
                error,even,exponent,fail,filter,flip,floatDigits,floatRadix,floatRange,floor,
                fmap,foldl,foldl1,foldr,foldr1,fromDouble,fromEnum,fromInt,fromInteger,fromIntegral,
                fromRational,gcd,getChar,getContents,getLine,head,id,inRange,index,init,intToDigit,
                interact,ioError,isAlpha,isAlphaNum,isAscii,isControl,isDenormalized,isDigit,isHexDigit,
                isIEEE,isInfinite,isLower,isNaN,isNegativeZero,isOctDigit,isPrint,isSpace,isUpper,iterate,iter,iterS,
                last,lcm,length,lex,lexDigits,lexLitChar,lines,log,logBase,lookup,map,mapM,mapM_,max,
                maxBound,maximum,maybe,min,minBound,minimum,mod,negate,not,notElem,null,numerator,odd,
                or,ord,pi,pred,primExitWith,print,product,properFraction,putChar,putStr,putStrLn,quot,
                quotRem,range,rangeSize,read,readDec,readFile,readFloat,readHex,readIO,readInt,readList,readLitChar,
                readLn,readOct,readParen,readSigned,reads,readsPrec,realToFrac,recip,refold,rem,replicate,
                reverse,round,scaleFloat,scanl,scanl1,scanr,scanr1,seq,sequence,sequence_,show,showChar,showInt,
                showList,showLitChar,showParen,showSigned,showString,shows,showsPrec,significand,signum,sin,
                sinh,snd,span,splitAt,sqrt,subtract,succ,sum,tail,take,takeWhile,tan,tanh,threadToIOResult,toEnum,
                toInt,toInteger,toLower,toRational,toUpper,truncate,uncurry,unlines,until,unwords,unzip,
                unzip3,userError,words,writeFile,zip,zip3,zipWith,zipWith3,listArray,doParse,lift,signal,get,put,extrude,return
        },
        emphstyle={[1]\textbf},
        emph=
        {[2]
                Double,Either,IO,Ordering,Rational,Ratio,ReadS,ShowS,
                Word8,InPacket,ReT,StT,I
        },
        emphstyle={[2]\textbf},
        emph=
        {[3]
                case,class,deriving,do,else,if,import,in,infixl,infixr,instance,let,
                module,of,primitive,then,data,type,vhdl,newtype
        },
        emphstyle={[3]\textbf},
        emph=
        {[4]
                quot,rem,div,mod,elem,notElem,seq
        },
%        emphstyle={[4]\color{castanho_ulisses}\textbf},
        emphstyle={[4]\textbf},
        emph=
        {[5]
                False,Just,Left,Nothing,Right,Show,Eq,Ord,Num
        },
%        emphstyle={[5]\color{preto_ulisses}\textbf}
        emphstyle={[5]\textbf}
}

\lstnewenvironment{footcode}
}
{}
%{\hrulesmallskip} not sure what this is.

%\lstnewenvironment{hcode}
%{%\textbf{Haskell Code} \hspace{1cm} \hrulefill
% \lstset{language=HaskellUlisses,    basicstyle=\ttfamily\small,escapechar=\%}}
%{}

\lstnewenvironment{hcode}
}
{}

\lstnewenvironment{smallcode}
{%\hspace{1cm}
 \lstset{language=HaskellUlisses,       basicstyle=\ttfamily\small,escapechar=\%
}}
{}
%{\hrulesmallskip} not sure what this is.

\lstnewenvironment{tinycode}
{%\hspace{1cm}
 \lstset{language=HaskellUlisses,       basicstyle=\ttfamily\scriptsize,escapechar=\%
}}
{}

\lstnewenvironment{teenycode}
{%\hspace{1cm}
 \lstset{language=HaskellUlisses,       basicstyle=\ttfamily\tiny,escapechar=\%
}}
{}


\lstnewenvironment{newcode}{\lstset{language=Haskell,basicstyle=\scriptsize,identifierstyle=\itshape,stringstyle=\itshape,escapechar=\%}}{}

\lstnewenvironment{normalcode}{\lstset{language=Haskell,
                                       basicstyle=\normalsize\ttfamily,
                                       keywordstyle=\bfseries,
                                       %identifierstyle=\itshape,
                                       %stringstyle=\itshape,
                                       escapechar=\%}}{}

\lstnewenvironment{vhdl}
{\lstset{language=VHDL,basicstyle=\ttfamily\small,escapechar=\%}}{}

\lstdefinelanguage{PreHDL} {
        basicstyle=\ttfamily\small,
        sensitive=true,
        morecomment=[l][\ttfamily\tiny]{//},
%        morecomment=[s][\ttfamily\tiny]{\{-}{-\}},
%       morecomment=[l][\color{gray_ulisses}\ttfamily]{--},
%       morecomment=[s][\color{gray_ulisses}\ttfamily]{\{-}{-\}},
%       morecomment=[l][\color{gray_ulisses}\ttfamily\tiny]{--},
%       morecomment=[s][\color{gray_ulisses}\ttfamily\tiny]{\{-}{-\}},
        morestring=[b]",
        %stringstyle=\color{red},
        showstringspaces=false,
        numberstyle=\tiny,
        numberblanklines=true,
        showspaces=false,
        breaklines=true,
        showtabs=false,
        emph=
        {[1]
                input,output,vhdl,label,if,else,yield,goto,boolean,bits,true,false,function,return,concat
        },
        emphstyle={[1]\textbf}
}

\lstnewenvironment{prehdl}
{\lstset{language=PreHDL,basicstyle=\ttfamily\small,escapechar=\%}}{}

\newcommand{\ra}{\ensuremath{\rightarrow}}
\newcommand{\edit}[1]{\marginpar{\raggedright\tiny\it{#1}}}

\newcommand{\incode}[1]{{\footnotesize{\tt{#1}}}}


\newcommand{\modal}[2]{\ensuremath{\mathsf{#1}_{#2}}}
\newcommand{\kripke}[1]{\ensuremath{\sim_{#1}}}
\renewcommand{\implies}{\ensuremath{\supset}}
\newcommand{\hbind}[1]{\ensuremath{\mbox{{\footnotesize{\texttt{{>}{>}{=}}}}}_{\mbox{\tiny {\it #1}}}}}
\newcommand{\hbindnull}[1]{\ensuremath{\footnotesize{\texttt{{>}{>}}}_{\mbox{\tiny {\it #1}}}}}
\newcommand{\hreturn}[1]{\ensuremath{\mathtt{return}_{\mbox{\tiny {\it #1}}}}}
%\renewcommand{\code}[1]{\ensuremath{\mathtt{#1}}}
\newcommand{\tyjudgment}[3]{\ensuremath{#1 \triangleright #2 : #3 }}
\newcommand{\nil}{\ensuremath{\mathtt{()}}}
\newcommand{\hmodels}{\tiny{\ddtstile{}{}}}
\newcommand*\widebar[1]{%
  \hbox{%
    \vbox{%
      \hrule height 0.5pt % The actual bar
      \kern0.1ex%         % Distance between bar and symbol
      \hbox{%
        \kern-0.1em%      % Shortening on the left side
        \ensuremath{#1}%
        \kern-0.1em%      % Shortening on the right side
      }%
    }%
  }%
} 

\newcommand{\parS}{\ensuremath{{{.}{\mathtt{\&}}{.}}}}
\newcommand{\parA}{\ensuremath{{{.}{\ast}{.}}}}
%\newcommand{\pardev}{\ensuremath{{{\mathtt{<}}{\mathtt{\&}}{\mathtt{>}}}}}
\newcommand{\pardev}{\ensuremath{\mbox{{\texttt{<\&>}}}}}


%\usetheme{Singapore}
%\useoutertheme{myinfolines}
\usecolortheme{rose}

%\usecolortheme{seahorse}
\useinnertheme[shadow]{rounded} 
\setbeamertemplate{navigation symbols}{}

\usepackage[all]{xy}
\usepackage{xmpmulti}
%\usepackage{pdfpages}

\newenvironment{codeblock}[1][.8]{%
\begin{columns}
\begin{column}{#1\linewidth}
\begin{exampleblock}{}}{%
\end{exampleblock}
\end{column}
\end{columns}} 

\newenvironment{execblock}[1][.8]{%
\begin{columns}
\begin{column}{#1\linewidth}
\begin{block}{}}{%
\end{block}
\end{column}
\end{columns}} 

\def\slideskip{\vskip 0.1in}
\def\frameskip{\vskip 0.1in}


%\begin{frame}
%\begin{figure}[t]
%\includegraphics[width=1\textwidth]{example1.png}
%\end{figure}
%\end{frame}

\title[CS4450]{CS4450/7450\\Chapter 2: Starting Out}
\subtitle{Principles of Programming Languages}
\author[Bill Harrison]{Dr. William Harrison}
\institute{University of Missouri}
%\date{August 31, 2011}

\begin{document}

\frame{\titlepage}

\begin{frame}[fragile]
\frametitle{GHCi is basically a fancy calculator}

\begin{codeblock}
\begin{hcode}
$ ghci
GHCi, version 7.10.3: http://www.haskell.org/ghc/  :? for help
Prelude> 4 + 2
6
Prelude> not (True && True)
False
Prelude> max 5 4
5
\end{hcode}
\end{codeblock}
\end{frame}

\begin{frame}[fragile]
\frametitle{Type errors are your friends}
\begin{codeblock}
\begin{hcode}
Prelude> 99 + "Hey"
<interactive>:5:4:
    No instance for (Num [Char]) arising from a use of `+`
    In the expression: 99 + "Hey"
    In an equation for `it`: it = 99 + "Hey"
Prelude> 
\end{hcode}
\end{codeblock}

\end{frame}

\begin{frame}[fragile]
\frametitle{GHCi Commands}
\framesubtitle{Some Pragmatics}
\begin{itemize}
\item \verb+:l+ or \verb+:load+ --- load a file or module

\item \verb+:t:+ or \verb+:type+ --- give the type of an expression

\item \verb+:i+ or \verb+:info+ --- produce information about a definition

\item \verb+:q+ or \verb+:quit+ --- quit, derp.

\end{itemize}

\pause
\begin{codeblock}
\begin{hcode}
Prelude> :t not
not :: Bool -> Bool
Prelude> :i not
not :: Bool -> Bool 	-- Defined in ?GHC.Classes?
Prelude> 
\end{hcode}
\end{codeblock}

\end{frame}


\begin{frame}[fragile]
\frametitle{``Baby's First Program''}

Entered in a file \verb+Chap2.hs+:
\begin{codeblock}
\begin{hcode}
module Chap2 where

doubleMe x = x + x  
\end{hcode}
\end{codeblock}

\end{frame}

\begin{frame}[fragile]
\frametitle{``Baby's First Program'', cont'd}
\begin{codeblock}
\begin{hcode}
$ ghci
GHCi, version 7.10.3: http://www.haskell.org/ghc/  :? for help
Prelude> :l Chap2.hs
[1 of 1] Compiling Chap2            ( Chap2.hs, interpreted )
Ok, modules loaded: Chap2.
*Chap2> doubleMe 9
18
*Chap2> doubleMe 3.14
6.28
*Chap2> :t doubleMe
\end{hcode}
\end{codeblock}
\pause
\begin{codeblock}
\begin{hcode}
doubleMe :: Num a => a -> a
*Chap2> 
\end{hcode}
\end{codeblock}
\end{frame}


\begin{frame}[fragile]
\frametitle{Lists, an Introduction to}
\begin{codeblock}
\begin{hcode}
Prelude> let lostNumbers = [4,8,15,16,23,42] 
Prelude> lostNumbers
[4,8,15,16,23,42]

Prelude> 99 : lostNumbers
[99,4,8,15,16,23,42]

Prelude> [1,2,3,4] ++ [9,10,11,12]
[1,2,3,4,9,10,11,12]

Prelude> "hello" ++ " " ++ "world"
"hello world"

Prelude> ['w','0'] ++ ['0','t']
"w00t"
\end{hcode}
\end{codeblock}
\end{frame}

\begin{frame}[fragile]
\Large

\frametitle{Type Declarations}

In Haskell, a new name for an existing type can be 
defined using a \underline{type declaration}. 

\slideskip

\begin{codeblock}
\begin{verbatim}
type String = [Char] 
\end{verbatim}
\end{codeblock}

\slideskip

{\tt String} is a synonym for the type \verb+[Char]+.
\end{frame}


\begin{frame}[fragile]
\Large

%\frametitle{Type Declarations}

Type declarations can be used to make other types 
easier to read.  
For example, given 

\slideskip

\begin{codeblock}
\begin{verbatim}
type Pos = (Int,Int)
\end{verbatim}
\end{codeblock}

\slideskip
we can define
\slideskip
\begin{codeblock}
\begin{verbatim}
origin    :: Pos 
origin     = (0,0) 

left      :: Pos -> Pos 
left (x,y) = (x-1,y) 
\end{verbatim}
\end{codeblock}

\end{frame}

\begin{frame}[fragile]
\Large

Like function definitions, type declarations can also 
have \underline{parameters}. 
For example, given 

\slideskip

\begin{codeblock}
\begin{verbatim}
type Pair a = (a,a) 
\end{verbatim}
\end{codeblock}

\slideskip
we can define
\slideskip

\begin{codeblock}
\begin{verbatim}
mult      :: Pair Int -> Int 
mult (m,n) = m*n

copy      :: a -> Pair a 
copy x     = (x,x)
\end{verbatim}
\end{codeblock}

\end{frame}

\begin{frame}[fragile]
\Large

Type declarations can be nested:

\slideskip

\begin{codeblock}
\begin{verbatim}
type Pos   = (Int,Int)    -- GOOD

type Trans = Pos -> Pos   -- GOOD
\end{verbatim}
\end{codeblock}

\slideskip
However, they cannot be recursive:
\slideskip

\begin{codeblock}
\begin{verbatim}
type Tree = (Int,[Tree])  -- BAD
\end{verbatim}
\end{codeblock}

\end{frame}

\begin{frame}[fragile]
\LARGE

\frametitle{Data Declarations}

A completely new type can be defined by specifying 
its values using a \underline{data declaration}. 

\slideskip

\begin{codeblock}
\begin{verbatim}
data Bool = False | True
\end{verbatim}
\end{codeblock}

\slideskip
{\tt Bool} is a new type, with two 
new values {\tt False} and {\tt True}.

\end{frame}

\begin{frame}[fragile]
\LARGE

Note: 
\begin{itemize}
\item The two values False and True are called the 
constructors for the type Bool. 
\item Type and constructor names must begin with 
an upper-case letter. 
\item Data declarations are similar to context free 
grammars.  The former specifies the values of 
a type, the latter the sentences of a language.
\end{itemize}
\end{frame}


\begin{frame}[fragile]
\large

Values of new types can be used in the same ways 
as those of built in types.  For example, given
\slideskip

\begin{codeblock}
\begin{verbatim}
data Answer = Yes | No | Unknown 
\end{verbatim}
\end{codeblock}

\slideskip
we can define:
\slideskip

\begin{codeblock}
\begin{verbatim}
answers     :: [Answer] 
answers      = [Yes,No,Unknown] 

flip        :: Answer -> Answer 
flip Yes     = No 
flip No      = Yes 
flip Unknown = Unknown 
\end{verbatim}
\end{codeblock}

\end{frame}

\begin{frame}[fragile]
\large

The constructors in a data declaration can also have 
parameters.  For example, given 
\slideskip

\begin{codeblock}
\begin{verbatim}
data Shape = Circle Float 
           | Rect Float Float 
\end{verbatim}
\end{codeblock}

\slideskip
we can define:
\slideskip

\begin{codeblock}
\begin{verbatim}
square         :: Float -> Shape 
square n        = Rect n n 
area           :: Shape -> Float 
area (Circle r) = pi * r^2 
area (Rect x y) = x * y 
\end{verbatim}
\end{codeblock}

\end{frame}

\begin{frame}[fragile]
\LARGE

Note: 
\begin{itemize}
\item {\tt Shape} has values of the form {\tt Circle r} where {\tt r} is 
a float, and {\tt Rect x y} where {\tt x} and {\tt y} are floats. 
\item {\tt Circle} and {\tt Rect} {\bf are} functions that 
construct values of type {\tt Shape}: 
\end{itemize}


\begin{execblock}[1.0]
\begin{verbatim}
-- Not a definition
Circle :: Float -> Shape 
Rect   :: Float -> Float -> Shape
\end{verbatim}
\end{execblock}


\end{frame}



\begin{frame}[fragile]
\large

Not surprisingly, data declarations themselves can 
also have parameters.  For example, given 
\slideskip

\begin{codeblock}
\begin{verbatim}
data Maybe a = Nothing | Just a 
\end{verbatim}
\end{codeblock}

\slideskip
we can define:
\slideskip

\begin{codeblock}
\begin{verbatim}
safediv    :: Int -> Int -> Maybe Int 
safediv _ 0 = Nothing 
safediv m n = Just (m `div` n) 

safehead   :: [a] -> Maybe a 
safehead [] = Nothing 
safehead xs = Just (head xs) 
\end{verbatim}
\end{codeblock}

\end{frame}

\begin{frame}[fragile]
\Large

\frametitle{Recursive Types}

In Haskell, new types can be declared in terms of 
themselves.  That is, types can be \underline{recursive}. 

\slideskip

\begin{codeblock}
\begin{verbatim}
data Nat = Zero | Succ Nat
\end{verbatim}
\end{codeblock}

\slideskip
Nat is a new type, with constructors 
{\tt Zero :: Nat} and {\tt Succ :: Nat -> Nat}.

\end{frame}

\begin{frame}[fragile]
\LARGE

Note: 
\begin{itemize}
\item A value of type {\tt Nat} is either {\tt Zero}, or of the form 
{\tt Succ n} where {n :: Nat}.  That is, {\tt Nat} contains the 
following infinite sequence of values: 
\end{itemize}

\begin{execblock}
\begin{verbatim}
Zero
\end{verbatim}
\end{execblock}
\begin{execblock}
\begin{verbatim}
Succ Zero
\end{verbatim}
\end{execblock}
\begin{execblock}
\begin{verbatim}
Succ (Succ Zero)
\end{verbatim}
\end{execblock}

\hskip 1in {\Huge$\vdots$}

\end{frame}


\begin{frame}[fragile]
\LARGE

Note: 
\begin{itemize}
\item We can think of values of type Nat as natural 
numbers, where {\tt Zero} represents {\tt 0}, and {\tt Succ} 
represents the successor function {\tt 1+}. 
\item For example, the value
\begin{execblock}
\begin{verbatim}
Succ (Succ (Succ Zero))
\end{verbatim}
\end{execblock}

represents the natural number
\begin{execblock}
\begin{verbatim}
1 + (1 + (1 + 0))
\end{verbatim}
\end{execblock}

\end{itemize}

\end{frame}



\begin{frame}[fragile]

\large
Using recursion, it is easy to define functions that 
convert between values of type Nat and Int: 

\frameskip

\begin{codeblock}
\begin{verbatim}
nat2int         :: Nat -> Int 
nat2int Zero     = 0 
nat2int (Succ n) = 1 + nat2int n 

int2nat         :: Int -> Nat 
int2nat 0        = Zero 
int2nat n        = Succ (int2nat (n - 1))
\end{verbatim}
\end{codeblock}

\end{frame}

\begin{frame}[fragile]

\large
Two naturals can be added by converting them to 
integers, adding, and then converting back: 

\frameskip

\begin{codeblock}
\begin{verbatim}
add    :: Nat -> Nat -> Nat 
add m n = int2nat (nat2int m + nat2int n) 
\end{verbatim}
\end{codeblock}
\frameskip
However, using recursion the function add can be 
defined without the need for conversions: 
\frameskip
\begin{codeblock}
\begin{verbatim}
add Zero     n = n 
add (Succ m) n = Succ (add m n) 
\end{verbatim}
\end{codeblock}
\frameskip
The recursive definition for add corresponds to 
the laws 
$$0+n = n$$ and $$(1+m)+n = 1+(m+n)$$
\end{frame}

\begin{frame}[fragile]

\Large
Using recursion, an expression tree can be defined using:

\frameskip

\begin{codeblock}
\begin{verbatim}
data Expr = Val Int 
          | Add Expr Expr 
          | Mul Expr Expr 
\end{verbatim}
\end{codeblock}
\frameskip

One example of such a tree written in Haskell is
\frameskip
\begin{execblock}
\begin{verbatim}
Add (Val 1) (Mul (Val 2) (Val 3))
\end{verbatim}
\end{execblock}
\frameskip


\end{frame}


\begin{frame}[fragile]

\Large
Using recursion, it is now easy to define functions 
that process expressions.  For example: 

\frameskip

\begin{codeblock}
\begin{verbatim}
size          :: Expr -> Int 
size (Val n)   = 1 
size (Add x y) = size x + size y 
size (Mul x y) = size x + size y 

eval          :: Expr -> Int 
eval (Val n)   = n 
eval (Add x y) = eval x + eval y 
eval (Mul x y) = eval x * eval y
\end{verbatim}
\end{codeblock}


\end{frame}

\begin{frame}[fragile]
\LARGE

Note: 
\begin{itemize}
\item 
The three constructors have types: 

\begin{execblock}[1.0]
\begin{verbatim}
-- Not a definition
Val :: Int -> Expr 
Add :: Expr -> Expr -> Expr 
Mul :: Expr -> Expr -> Expr 
\end{verbatim}
\end{execblock}

\end{itemize}

\end{frame}


\begin{frame}[fragile]

\Large
Using recursion, a binary tree can be defined using:

\frameskip

\begin{codeblock}
\begin{verbatim}
data Tree = Leaf Int 
          | Node Tree Int Tree 
\end{verbatim}
\end{codeblock}
\frameskip

One example of such a tree written in Haskell is
\frameskip
\begin{execblock}
\begin{verbatim}
Node (Node (Leaf 1) 3 (Leaf 4)) 
     5 
     (Node (Leaf 6) 7 (Leaf 9))
\end{verbatim}
\end{execblock}
\frameskip
\end{frame}


\begin{frame}[fragile]
\large

We can now define a function that decides if a given 
integer occurs in a binary tree: 

\frameskip

\begin{execblock}[1.0]
\begin{verbatim}
occurs               :: Int -> Tree -> Bool 
occurs m (Leaf n)     = m==n 
occurs m (Node l n r) = m==n 
                        || occurs m l 
                        || occurs m r 
\end{verbatim}
\end{execblock}

\frameskip

In the worst case, when the integer does not 
occur, this function traverses the entire tree.

\end{frame}

\begin{frame}[fragile]
\large

Search trees have the important property that when 
trying to find a value in a tree we can always decide 
which of the two sub-trees it may occur in: 

\frameskip

\begin{execblock}[1.0]
\begin{verbatim}
occurs               :: Int -> Tree -> Bool 
occurs m (Leaf n)            = m==n 
occurs m (Node l n r) | m==n = True 
                      | m<n  = occurs m l 
                      | m>n  = occurs m r
                      \end{verbatim}
\end{execblock}

\frameskip

This new definition is more \underline{efficient}, because it only 
traverses one path down the tree. 

What is the precondition for {\tt Node}?

\end{frame}
\begin{frame}[fragile]
\large

Finally consider the function {\tt flatten} that returns the 
list of all the integers contained in a tree: 

\frameskip

\begin{execblock}
\begin{verbatim}
flatten             :: Tree -> [Int] 
flatten (Leaf n)     = [n] 
flatten (Node l n r) = flatten l 
                       ++ [n] 
                       ++ flatten r 
\end{verbatim}
\end{execblock}

\frameskip
\frameskip

A tree is a \underline{search tree} if it flattens to a list that is ordered.
\end{frame}


\end{document}
